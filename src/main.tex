%----------------------------------------------------------------------------------------
%	PACKAGES AND OTHER DOCUMENT CONFIGURATIONS
%----------------------------------------------------------------------------------------

\documentclass[
	11pt, english, singlespacing,
%raft, % Uncomment to enable draft mode (no pictures, no links, overfull hboxes indicated)
%nolistspacing, % If the document is onehalfspacing or doublespacing, uncomment this to set spacing in lists to single
	liststotoc, % Uncomment to add the list of figures/tables/etc to the table of contents
	toctotoc, % Uncomment to add the main table of contents to the table of contents
%parskip, % Uncomment to add space between paragraphs
%nohyperref, % Uncomment to not load the hyperref package
%headsepline, % Uncomment to get a line under the header
%chapterinoneline, % Uncomment to place the chapter title next to the number on one line
%consistentlayout, % Uncomment to change the layout of the declaration, abstract and acknowledgements pages to match the default layout
]{MastersDoctoralThesis}

%----------------------------------------------------------------------------------------
%	PACKAGES
%----------------------------------------------------------------------------------------
\usepackage[utf8]{inputenc} % Required for inputting international characters
\usepackage[T1]{fontenc}
\usepackage{mathpazo} % Use the Palatino font by default


%-------------------------------- Define the LIBRARY ------------------------------------

\usepackage[backend=biber,style=numeric,sorting=none,sortcites=true,style=numeric-comp]{biblatex}
\AtBeginBibliography{\renewcommand*{\finalnamedelim}{\addspace\bibstring{and}\space}}
\renewcommand*{\bibnamedash}{\ifnumgreater{\value{liststop}}{1}{\space et al.}{}} % Et al. bei mehreren Autoren
%\renewcommand*{\bibfont}{\small} % Optional: Make the bibliography font size smaller
\addbibresource{/home/leopold/PycharmProjects/Master_thesis/out/My_bibliography.bib}
\RequireBibliographyStyle{customstyle_bibiliography} % Load the custom style
%----------------------------------------------------------------------------------------


\usepackage[autostyle=true]{csquotes}

\usepackage{hyperref}
\usepackage{amsmath, amssymb, amsfonts}
\usepackage{bm}       % For bold Greek letters
\usepackage{caption}
\captionsetup{width=1.0\textwidth, font=normal, labelfont=bf}

%----------------------------------------------------------------------------------------
%	MARGIN SETTINGS
%----------------------------------------------------------------------------------------

\geometry{
	paper=a4paper,
	inner=2cm, % Inner margin
	outer=3cm, % Outer margin
	bindingoffset=.5cm, % Binding offset
	top=1.5cm, % Top margin
	bottom=1.5cm, % Bottom margin
	%showframe, % Uncomment to show how the type block is set on the page
}

%----------------------------------------------------------------------------------------
%	THESIS INFORMATION
%----------------------------------------------------------------------------------------

\thesistitle{Master thesis}
\supervisor{\href{https://ic1.ugr.es/members/dmanzano/home/}{Prof. Dr. Daniel Manzano Diosdado}}
\examiner{Prof. Dr. Beatriz \textsc{Olmos Sanchez}}
\degree{Master of Science}
\author{Leopold \textsc{Bodamer}}
\addresses{Wolfsbergallee 17b, 75177 Pforzheim}
\subject{Theoretical Atomic Physics and Synthetic Quantum Systems}
\universityone{\href{https://uni-tuebingen.de\\}{Eberhard Karls Universität Tübingen}}
\universitytwo{\href{https://www.ugr.es\\}{Universidad de Granada}}

\department{\href{https://uni-tuebingen.de/fakultaeten/mathematisch-naturwissenschaftliche-fakultaet/fachbereiche/physik/institute/institut-fuer-theoretische-physik/arbeitsgruppen/}{Institut für Theoretische Physik}}
\group{\href{https://uni-tuebingen.de/fakultaeten/mathematisch-naturwissenschaftliche-fakultaet/fachbereiche/physik/institute/institut-fuer-theoretische-physik/arbeitsgruppen/ag-lesanovsky/}{Theoretical Atomic Physics and Synthetic Quantum Systems}}
\faculty{\href{https://uni-tuebingen.de/fakultaeten/mathematisch-naturwissenschaftliche-fakultaet/fakultaet/}{Mathematisch-Naturwissenschaftliche Fakultät}}

\AtBeginDocument{
\hypersetup{pdftitle=\ttitle}
\hypersetup{pdfauthor=\authorname}
}

\begin{document}

\frontmatter
\pagestyle{plain}

%----------------------------------------------------------------------------------------
%	TITLE PAGE
%----------------------------------------------------------------------------------------

\begin{titlepage}
  \begin{center}
  
  %\vspace*{.06\textheight}
  {\scshape\LARGE \univname\par}\vspace{1.5cm} % University name
  %\includegraphics[scale = 0.15]{Uni_logo}\vspace{1.5cm} % University/department logo - uncomment to place it
  
  \textsc{\Large Bachelor Thesis}\\[0.5cm] % Thesis type
  
  \HRule \\[0.4cm] % Horizontal line
  {\huge \bfseries \ttitle\par}\vspace{0.4cm} % Thesis title
  
  \HRule \\[1.5cm] % Horizontal line
   
  \begin{minipage}[t]{0.4\textwidth}
  \begin{flushleft} \large
  \emph{Author:}\\
  {\authorname} % Author name
  \end{flushleft}
  \end{minipage}
  \begin{minipage}[t]{0.4\textwidth}
  \begin{flushright} \large
  \emph{Supervisor:} \\
  \href{https://sites.google.com/site/beatrizolmosphysics/}{\supname} % Supervisor name - remove the \href bracket to remove the link  
  \end{flushright}
  \end{minipage}\\[3cm]
   
  \vfill
  
  \large \textit{A thesis submitted in fulfillment of the requirements\\ for the degree of \degreename}\\[0.3cm] % University requirement text
  \textit{in }\\[0.4cm]
  \groupname\\\deptname\\[2cm] % Research group name and department name
   
  \vfill
  
  {\large \today}\\[4cm] % Date
  
  \vfill
  \end{center}
  \end{titlepage}
  
  %----------------------------------------------------------------------------------------
  %	DECLARATION PAGE
  %----------------------------------------------------------------------------------------
  
  \begin{declaration}
  \addchaptertocentry{\authorshipname} % Add the declaration to the table of contents
  \noindent I, \authorname, declare that this thesis titled, \enquote{\ttitle} and the work presented in it are my own. I confirm that:
  
  \begin{itemize} 
  \item This work was done wholly or mainly while applying for a research degree at this University.
  \item Where any part of this thesis has previously been submitted for a degree or any other qualification at this University or any other institution, this has been clearly stated.
  \item Where I have consulted the published work of others, this is always clearly attributed.
  \item Where I have quoted from the work of others, the source is always given. With the exception of such quotations, this thesis is entirely my own work.
  \item I have acknowledged all main sources of help.
  \item Where the thesis is based on work done by myself jointly with others, I have made clear exactly what was done by others and what I have contributed myself.
  \end{itemize}
   
  \noindent Signed:\\
  \rule[0.5em]{25em}{0.5pt} % This prints a line for the signature
   
  \noindent Date:\\
  \rule[0.5em]{25em}{0.5pt} % This prints a line to write the date
  \end{declaration}

%----------------------------------------------------------------------------------------
%	ABSTRACT PAGE
%----------------------------------------------------------------------------------------

%\begin{abstract}
%\addchaptertocentry{\abstractname}
%This thesis investigates the directional routing of excitations in atomic systems using subradiant states.
%Building on Bottarelli's quantum router \cite{Startingpoint}, this thesis adapts the model to atomic systems,
%addressing the challenges of controlling interactions in fully connected systems.
%Atom light interactions have been heavily studied for atomic lattices \cite{Masson2022, Asenjo-Garcia2017, Needham2019, Cech2023, Jen2016}.
%In this theis three chains, that are connected by an equilateral triangle and an isosceles triangle are studied.
%By allowing different dipole orientations on each chain, three distinct topologies are considered.
%The results show that controlling the topology and the initial state enables directional routing,
%where a topology with equilateral triangle and aligned dipoles emerges as the most practical for stable readout.

	
%\end{abstract}
%----------------------------------------------------------------------------------------
%	CONTENTS
%----------------------------------------------------------------------------------------
\mainmatter
\pagestyle{thesis}

%%----------------------------------------------------------------------------------------
%	ACKNOWLEDGEMENTS
%----------------------------------------------------------------------------------------

\begin{acknowledgements}
\addchaptertocentry{\acknowledgementname}
I would like to express my deepest gratitude to those who supported me throughout this journey.
First and foremost, I sincerely thank my supervisor \supname for her guidance and encouragement.
Her mentorship was crucial in shaping this work.

\noindent
A special thanks to Marcel Cech, whose expertise and advice greatly contributed to the development of this work.
I am also deeply grateful for my colleagues and fellow students,
Niklas Schmid, Paul Haffner, Ardit Thaqi, Christian Gommeringer for their assistance,
which made this experience both productive and enjoyable.
One of them was Richard Marquardt, who was always ready to provide mathematical help.

\noindent
I would like to especially thank Malik Jirasek, Niclas Schilling, Anna Schupeta and Paul Haffner for proofreading this thesis.
Additionally,
I am thankful to Tabea Bodamer for her assistance with English grammar and to Konstanze Bodamer for her financial support.
\noindent
Finally, I would like to extend my appreciation to my girlfriend, Anna Schupeta, for her constant love, patience,
and understanding through this process.

\end{acknowledgements}

%----------------------------------------------------------------------------------------
%	LIST OF CONTENTS/FIGURES/TABLES PAGES
%----------------------------------------------------------------------------------------

\tableofcontents % Prints the main table of contents

%\listoffigures % Prints the list of figures
%\listoftables % Prints the list of tables

%----------------------------------------------------------------------------------------
%	ABBREVIATIONS
%----------------------------------------------------------------------------------------

%\begin{abbreviations}{ll} % Include a list of abbreviations (a table of two columns)
%
%\textbf{LAH} & \textbf{L}ist \textbf{A}bbreviations \textbf{H}ere\\
%\textbf{WSF} & \textbf{W}hat (it) \textbf{S}tands \textbf{F}
%
%\end{abbreviations}

%----------------------------------------------------------------------------------------
%	PHYSICAL CONSTANTS/OTHER DEFINITIONS
%----------------------------------------------------------------------------------------

%\begin{constants}{lr@{${}={}$}l} % The list of physical constants is a three column table
%
%% The \SI{}{} command is provided by the siunitx package, see its documentation for instructions on how to use it
%
%Speed of Light & $c_{0}$ & \SI{2.99792458e8}{\meter\per\second} (exact)\\
%%Constant Name & $Symbol$ & $Constant Value$ with units\\
%
%\end{constants}

%----------------------------------------------------------------------------------------
%	SYMBOLS
%----------------------------------------------------------------------------------------

%\begin{symbols}{lll} % Include a list of Symbols (a three column table)
%
%$a$ & distance & \si{\meter} \\
%$P$ & power & \si{\watt} (\si{\joule\per\second}) \\
%%Symbol & Name & Unit \\
%
%\addlinespace % Gap to separate the Roman symbols from the Greek
%
%$\omega$ & angular frequency & \si{\radian} \\
%
%\end{symbols}














%----------------------------------------------------------------------------------------
%	DEDICATION
%----------------------------------------------------------------------------------------

%  \dedicatory{For/Dedicated to/To my\ldots}
\chapter{Introduction} % Main chapter title
\label{Chapter1} % Change X to a consecutive number; for referencing this chapter elsewhere, use \ref{ChapterX}

%----------------------------------------------------------------------------------------
%	SECTION 1
%----------------------------------------------------------------------------------------
\section{Motivation}
\noindent
Quantum computing, an exceptionally promising area of study in modern physics, offers a fundamentally new way to process and transmit information.
Applications of this, not yet scalable technology range from cryptography to drug research.
Quantum computers would especially outperform classical computers in quantum simulations of chemical / physical systems \cite{Eddins2022}.
Qubits, the analog to classical bits, for example,
photons or atoms play a crucial role in harnessing this quantum technology \cite{Ramakrishnan2023}.
The area which describes photons and their interaction with matter is quantum optics.
One essential goal of this filed is to build efficient and controllable interactions between photons and atoms.
A challenge for this is unwanted (spontaneous) emission, where photons are scattered into channels out of control.
This spontaneous emission hampers the development of quantum technologies, especially in quantum information processing.
%Quantum information processing stands at the forefront of modern technological advancement.
Subradiant states are a promising concept for this field of study.
These states appear when many emitters interact via light-mediated resonant dipole-dipole interactions
and inherit lifetimes magnitudes larger than that of a single emitter \cite{AsenjoGarcia2017}.
Insights into information transport within complex systems are of utmost interest,
as they could lead to advances in quantum computing. %routing and storage
Especially with subradiant states, as they also offer ultrafast readout \cite{Scully2015}.
%Zhen Wang et al. demonstrated, that a switching between sub- and superradiant modes is possible \cite{Wang2020}.

%----------------------------------------------------------------------------------------
%	SECTION 1
%----------------------------------------------------------------------------------------
%\section{Objective}
\vspace{0.5cm}
\noindent
The goal of this thesis is
to perform robust directional photon routing on atomic systems in free-space using subradiant states.
Focusing on a Y-shaped atomic tree, different topologies are explored to enable long-lived information transport as a proof of concept.

    \section{Outline}
This thesis is structured as follows.
Chapter \ref{Chapter2} introduces the theoretical background.
It covers the concepts of open quantum systems,
subradiance and superradiance, the Green tensor, and the reciprocal space.
These tools are essential foundations for describing atom-atom interactions in free space,
including dipole-dipole interactions and coupling to a photonic bath.
%After this chapter, the reader already knows...
The quantum router of \cite{Startingpoint} is presented and summarized in Chapter \ref{Chapter3}.
It introduces the concepts of graph theory and explains how quantum evolution on a graph topology can be utilized to achieve directional routing of information.
Chapter \ref{Chapter4} will be the core of this thesis, adapting this model to an atomic system.
This chapter delves into the challenges of implementing directional routing in a fully connected atomic system and investigates various solutions to control the phase of interactions.
It further extends the analysis to systems with a larger number of atoms, focusing on coupling control and routing capabilities in different configurations, such as equilateral and isosceles triangles.
Chapter \ref{Chapter5} concludes the thesis by summarizing the results and discussing potential future directions in the field of quantum routing in atomic systems.
%\include{Chapters/C2_Theoretical_Background}
%\include{Chapters/C3_Presenting_Basis}
%\include{Chapters/C4_Adaptation_to_atoms}
%\include{Chapters/C5_Conclusion}

%----------------------------------------------------------------------------------------
%	APPENDICES
%----------------------------------------------------------------------------------------

\appendix
%\include{Chapters/AppendixA}

%----------------------------------------------------------------------------------------
%	BIBLIOGRAPHY
%----------------------------------------------------------------------------------------

\printbibliography

\end{document}


% https://www.bing.com/images/search?view=detailV2&ccid=QOs8jh%2FY&id=B2F27160AAD4B68E1198585266845260137E1ED1&thid=OIP.QOs8jh_YfvbZty3qEQtmMwHaFv&mediaurl=https%3A%2F%2Fpubs.acs.org%2Fcms%2F10.1021%2Facs.jpcc.1c02693%2Fasset%2Fimages%2Flarge%2Fjp1c02693_0002.jpeg&cdnurl=https%3A%2F%2Fth.bing.com%2Fth%2Fid%2FR.40eb3c8e1fd87ef6d9b72dea110b6633%3Frik%3D0R5%252bE2BShGZSWA%26pid%3DImgRaw%26r%3D0&exph=1227&expw=1582&q=two+dimensional+spectroscopy+of+open+quantum+systems&simid=608020752819562198&form=IRPRST&ck=78FA3137380D45FD044E82D5CFD9637B&selectedindex=0&itb=0&ajaxhist=0&ajaxserp=0&vt=0&sim=11
%\begin{itemize}
%\item "Principles of Nonlinear Optical Spectroscopy" by Shan and Kirtman:
%\begin{itemize}
%\item This book provides a solid introduction to nonlinear optical spectroscopy, including 2D spectroscopy, and outlines methods for simulating spectroscopic signals from simple systems.
%\end{itemize}
%
%\item "Two-Dimensional Correlation Spectroscopy: Applications in Vibrational and Optical Spectroscopy" by R. M. B. L. McDonald and B. M. H. Lang:
%
%\begin{itemize}
%\item Focuses on 2D correlation spectroscopy and its applications in various spectroscopic fields, with simpler models used to illustrate the methods.
%\end{itemize}
%
%\item "Quantum Coherence in Systems of Atoms and Molecules: Theoretical and Experimental Studies" by A. Mukamel:
%
%\begin{itemize}
%\item This is a good resource for quantum descriptions of systems and how to simulate nonlinear spectroscopies like 2D spectroscopy in simple systems.
%\end{itemize}
%
%\item Review paper:
%
%\begin{itemize}
%\item \textit{"Two-Dimensional Electronic Spectroscopy"}, by A. M. Walmsley, Y. O. I. Shim, et al. (2009).\begin{itemize}
%\item A very detailed review of 2D electronic spectroscopy, including its application to simple systems, methods of analysis, and computational approaches.
%\end{itemize}
%
%\end{itemize}
%
%\end{itemize}
%\subsection*{2. Scientific Papers with Simulations of Simple Systems}
%\begin{itemize}
%\item "Two-dimensional electronic spectroscopy of molecular aggregates" by J. M. Thomas, D. V. Voronine, et al. (2013):
%
%\begin{itemize}
%\item Provides a step-by-step guide to performing 2D spectroscopy on simple systems like molecular aggregates. This paper discusses both the theoretical framework and simulations.
%\end{itemize}
%
%\item "Quantum Coherent Effects in Two-Dimensional Electronic Spectroscopy of Model Systems" by Y. Yan, A. Aspuru-Guzik, et al. (2010):
%
%\begin{itemize}
%\item Focuses on the theoretical treatment of 2D electronic spectroscopy of small systems, including simple chromophores and model molecular systems.
%\end{itemize}
%
%\item "Two-Dimensional Fourier Transform Spectroscopy of Single Chromophores" by D. V. Voronine, et al. (2012):
%
%\begin{itemize}
%\item The paper discusses the application of 2D spectroscopy to individual chromophores (a simple system) and explains how to extract physical information from the experimental data.
%\end{itemize}
%
%\end{itemize}
%\subsection*{3. Computational Methods and Software}
%If you're interested in simulating 2D electronic spectroscopy numerically for simple systems, the following computational tools are often used:
%
%\begin{itemize}
%\item PySCF (Python for Strongly Correlated Electron Systems):
%
%\begin{itemize}
%\item A computational chemistry package that can simulate molecular systems and might help you calculate the response functions used in 2D spectroscopy.
%\end{itemize}
%
%\item Qutip (Quantum Toolbox in Python):
%
%\begin{itemize}
%\item Useful for simulating open quantum systems, calculating time evolution, and modeling processes like those in 2D electronic spectroscopy.
%\end{itemize}
%
%\item TDSCF (Time-Dependent Self-Consistent Field Theory):
%
%\begin{itemize}
%\item A method often used in simulations of 2D electronic spectroscopy, this software may help in constructing a basic model for your system.
%\end{itemize}
%
%\item 2D NMR (for Nuclear Systems):
%
%\begin{itemize}
%\item While this focuses on nuclear magnetic resonance, many of the methods developed for 2D NMR spectroscopy can also be adapted to electronic spectroscopy, and relevant software tools include NMRPipe or MestReNova.
%\end{itemize}
%
%\end{itemize}
%\subsection*{4. Additional Resources}
%\begin{itemize}
%\item Tutorials and Lecture Notes:\begin{itemize}
%\item Several universities offer courses on nonlinear spectroscopy and 2D electronic spectroscopy. Look for lecture notes or course material available online, which can provide step-by-step methods to simulate simple systems. For example, the MIT OpenCourseWare site often has useful resources for quantum and optical spectroscopy.
%\end{itemize}
%
%\item Theoretical Approaches:\begin{itemize}
%\item "Nonlinear Optical Spectroscopy" by O. L. Chapman: This book will provide you with insights into the theory and mathematical foundations of 2D electronic spectroscopy for simple molecular systems.
%\item "Nonlinear Optics" by Robert W. Boyd: Boyd's book is a widely regarded source for nonlinear optics, including 2D spectroscopy, and covers applications to simpler systems such as isolated molecules or aggregates.
%\end{itemize}
%
%\end{itemize}

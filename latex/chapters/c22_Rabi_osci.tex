% Chapter Template

\chapter{Rabi Oscillations and Related Concepts} % Main chapter title

\label{ChapterRabiOscillations} % For referencing this chapter elsewhere, use \ref{ChapterRabiOscillations}

%-----------------------------------
%	SUBSECTION 2: Density Matrix Formalism
%-----------------------------------
\section{Introduction}
\subsection{Density Matrix Formalism}

The density matrix formalism provides a powerful framework to describe the dynamics of quantum systems, especially when dealing with mixed states or decoherence.
The density matrix \(\rho\) is defined as:
\begin{equation}
	\rho = |\psi\rangle \langle \psi|,
	\label{eq:DensityMatrix}
\end{equation}
for pure states, and as a statistical mixture for mixed states.
In this formalism, the diagonal elements represent populations and the off-diagonal elements represent coherences between states.
Coherences are phase relations between different quantum states, which are crucial for interference.
For example for a two level system a clear phase relation and a pure quantum state would have off diagonal elements of $ \rho_{ij} = 1/2$.
This state is often referred to as a coherent superposition of the two states.

When coupling a system to an environment, the environment is responsible for decoherence.
The state evolves over time to a purely statistical mixture of states.


The time evolution of the density matrix is governed by the Liouville-von Neumann equation:
\begin{equation}
	\frac{\partial \rho}{\partial t} = -\frac{i}{\hbar} [H, \rho],
	\label{eq:Liouville}
\end{equation}
where \(H\) is the system Hamiltonian.

In the presence of decoherence or dissipation, the dynamics can be described using the Lindblad master equation:
\begin{equation}
	\frac{\partial \rho}{\partial t} = -\frac{i}{\hbar} [H, \rho] + \sum_k \mathcal{L}_k(\rho),
	\label{eq:Lindblad}
\end{equation}
where \(\mathcal{L}_k(\rho)\) are Lindblad operators modeling the interaction with the environment.


%----------------------------------------------------------------------------------------
%	SECTION 1: Rabi Oscillations
%----------------------------------------------------------------------------------------

\section{Rabi Oscillations}

Rabi oscillations describe the coherent oscillatory behavior of a two-level quantum system interacting with a resonant electromagnetic field.
This phenomenon is fundamental in quantum mechanics and quantum optics, with applications in quantum computing, spectroscopy, and atomic physics.

\subsection{1. Schrödinger Picture}
Consider a two-level system with states \(|g\rangle\) (ground state) and \(|e\rangle\) (excited state).
The energy separation between the two states is given by:
\begin{equation}
	\omega_0 = \frac{E_e - E_g}{\hbar},
	\label{eq:EnergySeparation}
\end{equation}
where \(E_e\) and \(E_g\) are the energies of the excited and ground states, respectively.

The system Hamiltonian is expressed as:
\begin{equation}
	H_S = \hbar \omega_0 \ket{e}\bra{e}.
	\label{eq:SystemHamiltonian}
\end{equation}


The interaction of the system with a classical electromagnetic field \(E(t)\) is described by the time-dependent interaction Hamiltonian:
\begin{equation}
	H_{\text{int}}(t) = - \mu E(t) = -\left( \mu_{eg} \ket{e}\bra{g} + \mu_{ge} \ket{g}\bra{e} \right) E(t),
	\label{eq:InteractionHamiltonian}
\end{equation}
where \(\mu_{eg}\) and \(\mu_{ge}\) are the dipole matrix elements.

The total Hamiltonian of the system is then given by:
\begin{equation}
	H(t) = H_S + H_{\text{int}}(t),
	\label{eq:TotalHamiltonian}
\end{equation}
which combines the system's intrinsic energy and its interaction with the field.

Alternatively, the total Hamiltonian can be written in terms of Pauli matrices as:
\begin{equation}
	H(t) = \frac{\hbar \omega_0}{2} \sigma_z + \hbar \Omega \cos(\omega_L t + \phi) \sigma_x,
	\label{eq:RabiHamiltonian}
\end{equation}
where \(\Omega\) is the Rabi frequency, proportional to the field amplitude and the dipole matrix element, \(\phi\), \(\omega_L\) is the phase, frequency of the driving field respectively, and \(\sigma_z\) and \(\sigma_x\) are the Pauli matrices

The time evolution of the system is governed by the time dependent Schrödinger equation:
\begin{equation}
	i\hbar \frac{\partial}{\partial t} |\psi(t)\rangle = H(t) |\psi(t)\rangle,
	\label{eq:SchrodingerEquation}
\end{equation}
which describes the dynamics of the quantum state \(|\psi(t)\rangle\) under the influence of the Hamiltonian \(H(t)\).

%-----------------------------------
%	SUBSECTION 1: Rotating Wave Approximation (RWA)
%-----------------------------------

\subsection{Rotating Wave Approximation (RWA)}

The rotating wave approximation simplifies the analysis of the Hamiltonian in Eq. \eqref{eq:RabiHamiltonian}.
This approximation is valid when \(\Omega \ll \omega_0\), allowing us to focus on the resonant interaction.
The RWA reveals the essence of Rabi oscillations, where the population of the two levels oscillates with the Rabi frequency \(\Omega_R = \sqrt{\Delta^2 + \Omega^2}\).

\section{Applying the RWA Explicitly}

Suppose the electric field is classical and oscillatory:
\begin{equation}
	E(t) = E_0 \cos(\omega_L t + \phi) = \frac{E_0}{2}\left(e^{i\omega_L t + i \phi} + e^{-i\omega_L t - i \phi}\right)
	\label{eq:ElectricField}
\end{equation}

Assuming \(\mathbf{d} = \mathbf{\mu}_{eg} \ket{e}\bra{g} + \mathbf{d}_{ge} \ket{g}\bra{e}\) and defining the Rabi frequency:
\begin{equation}
	\hbar\Omega = -\mathbf{\mu}_{eg} \cdot \mathbf{E}_0
\end{equation}
is the Rabi frequency, proportional to the field amplitude and the dipole matrix element.

Then the interaction Hamiltonian \eqref{eq:InteractionHamiltonian} becomes:
\begin{equation}
	H_{\text{int}}(t) = \hbar\Omega \cos(\omega_L t + \phi) (\ket{e}\bra{g} + \ket{g}\bra{e})
	\label{eq:InteractionHamiltonian_RABI}
\end{equation}

Now go to the interaction picture with a unitary transformation \( U_0(t) \):
\begin{equation}
	U_0(t) = e^{-i H_0 t / \hbar} = e^{-i \omega_0 t \ket{e}\bra{e}}
	\label{eq:UnitaryTransformationH0_to_Interaction_pic}
\end{equation}

The interaction Hamiltonian in the interaction picture is with Eq. \eqref{eq:UnitaryTransformationH0_to_Interaction_pic}:
\begin{equation}
	H_{\text{int}}^{(I)}(t) = U_0^\dagger(t) H_{\text{int}}(t) U_0(t).
	\label{eq:InteractionHamiltonianInteractionPicture}
\end{equation}

Under this transformation, the operators evolve as described in Eq. \eqref{eq:InteractionHamiltonian_RABI}:
\begin{equation}
	\ket{e}\bra{g} \rightarrow e^{i\omega_0 t} \ket{e}\bra{g}, \quad
	\ket{g}\bra{e} \rightarrow e^{-i\omega_0 t} \ket{g}\bra{e}.
	\label{eq:OperatorEvolution}
\end{equation}

which results in
\begin{equation}
	H_{\text{int}}^{(I)}(t) = \hbar\Omega \cos(\omega_L t + \phi)\left(e^{i\omega_0 t} \ket{e}\bra{g} + e^{-i\omega_0 t} \ket{g}\bra{e} \right).
	\label{eq:InteractionHamiltonianTransformed}
\end{equation}

\subsection*{Rotating Frame Transformation}

We now define the unitary transformation:
\begin{equation}
	U_L(t) = e^{i \omega_L t \ket{e}\bra{e}} =
	\begin{pmatrix}
		1 & 0               \\
		0 & e^{i\omega_L t}
	\end{pmatrix}
	\label{eq:UnitaryTransformation}
\end{equation}
that rotates the reference frame of the system to the rotating frame at frequency \(\omega_L\).

The transformed density matrix is:
\begin{equation}
	\tilde{\rho}(t) = U_L^\dagger(t) \rho(t) U_L(t)
	\label{eq:TransformedDensityMatrix}
\end{equation}

and in this frame, the total transformed Hamiltonian is:
\begin{equation}
	\tilde{H}^I(t) = U_L^\dagger(t) H_{\text{int}}^I(t) U_L(t) - i\hbar U_L(t) \frac{d}{dt} U_L^\dagger(t)
	\label{eq:TransformedHamiltonian}
\end{equation}

The second term gives:
\begin{equation}
	-i\hbar U_L^\dagger(t) \frac{d}{dt} U_L(t) = - \hbar \omega_L \ket{e}\bra{e}
	\label{eq:SecondTermHamiltonian}
\end{equation}

Thus:
\begin{equation}
	\tilde{H}^I(t) = - \hbar \omega_L \ket{e}\bra{e} + U_L^\dagger(t) H_{\text{int}}^I(t) U_L(t)
	\label{eq:FinalTransformedHamiltonian}
\end{equation}


\subsection*{Transforming the Interaction Hamiltonian}

In the rotating frame:
\begin{equation}
	U_L^\dagger(t) \ket{e}\bra{g} U_L(t) = e^{-i\omega_L t} \ket{e}\bra{g}, \quad
	U_L^\dagger(t) \ket{g}\bra{e} U_L(t) = e^{i\omega_L t} \ket{g}\bra{e}
	\label{eq:RotatingFrameOperators}
\end{equation}

The interaction Hamiltonian becomes:
\begin{equation}
	\tilde{H}_{\text{int}}(t) = -\hbar\Omega \cos(\omega_L t + \phi)\left(e^{i(\omega_0 - \omega_L)t} \ket{e}\bra{g} + e^{-i(\omega_0 - \omega_L)t} \ket{g}\bra{e} \right)
	\label{eq:TransformedInteractionHamiltonian}
\end{equation}

\subsection*{Rotating Wave Approximation (RWA)}
We can use the identity:
\begin{equation}
	\cos(\omega_L t + \phi) = \frac{1}{2}\left(e^{i(\omega_L t + \phi)} + e^{-i(\omega_L t + \phi)} \right)
\end{equation}

Under RWA, drop fast-rotating terms \(e^{\pm i2\omega_L t}\), keeping only:
\begin{equation}
	\tilde{H}_{\text{RWA}} = -\hbar \Delta \ket{e}\bra{e} - \frac{E_0}{2} \left(
	\mu_{eg} \ket{e}\bra{g} + \mu_{ge} \ket{g}\bra{e}
	\right)
	\label{eq:RWAHamiltonianFinal}
\end{equation}

\subsection*{Equation of Motion}

The von Neumann equation becomes:
\begin{equation}
	\frac{d}{dt} \tilde{\rho}(t) = -\frac{i}{\hbar} [\tilde{H}_{\text{RWA}}, \tilde{\rho}(t)] + \text{(dissipation terms)}
	\label{eq:VonNeumannEquationRWA}
\end{equation}

To recover the entries of the original density matrix \(\rho(t)\) from the evolved density matrix in the rotating frame \(\tilde{\rho}(t)\), we use the inverse of the unitary transformation \(U_L(t)\):
\begin{equation}
	\rho(t) = U_L^\dagger(t) \tilde{\rho}(t) U_L(t)
	\label{eq:RecoverOriginalDensityMatrix}
\end{equation}
where \(U_L(t)\) and \(U_L^\dagger(t)\) are given in Eq. \eqref{eq:UnitaryTransformation}.

Let the density matrices be:
\begin{equation}
	\rho(t) =
	\begin{pmatrix}
		\rho_{gg}(t) & \rho_{ge}(t) \\
		\rho_{eg}(t) & \rho_{ee}(t)
	\end{pmatrix}, \quad
	\tilde{\rho}(t) =
	\begin{pmatrix}
		\tilde{\rho}_{gg}(t) & \tilde{\rho}_{ge}(t) \\
		\tilde{\rho}_{eg}(t) & \tilde{\rho}_{ee}(t)
	\end{pmatrix}
	\label{eq:DensityMatrices}
\end{equation}

The recovery process is:
\begin{equation}
	\rho(t) =
	\begin{pmatrix}
		1 & 0                \\
		0 & e^{-i\omega_L t}
	\end{pmatrix}
	\begin{pmatrix}
		\tilde{\rho}_{gg}(t) & \tilde{\rho}_{ge}(t) \\
		\tilde{\rho}_{eg}(t) & \tilde{\rho}_{ee}(t)
	\end{pmatrix}
	\begin{pmatrix}
		1 & 0               \\
		0 & e^{i\omega_L t}
	\end{pmatrix}
	\label{eq:RecoveryProcess}
\end{equation}

Simplifying:
\begin{equation}
	\rho(t) =
	\begin{pmatrix}
		\tilde{\rho}_{gg}(t)                  & e^{i\omega_L t} \tilde{\rho}_{ge}(t) \\
		e^{-i\omega_L t} \tilde{\rho}_{eg}(t) & \tilde{\rho}_{ee}(t)
	\end{pmatrix}
	\label{eq:FinalRecoveredDensityMatrix}
\end{equation}

Thus, the entries of the original density matrix $\rho(t)$ are related to the entries of the density matrix in the rotating frame $\tilde{\rho}(t)$ by:
\begin{align*}
	\rho_{gg}(t) & = \tilde{\rho}_{gg}(t)                  \\
	\rho_{ee}(t) & = \tilde{\rho}_{ee}(t)                  \\
	\rho_{ge}(t) & = e^{i\omega_L t} \tilde{\rho}_{ge}(t)  \\
	\rho_{eg}(t) & = e^{-i\omega_L t} \tilde{\rho}_{eg}(t)
\end{align*}



By moving to a rotating frame and neglecting rapidly oscillating terms, the effective Hamiltonian becomes:
\begin{equation}
	H_{\text{RWA}} = \frac{\hbar \Delta}{2} \sigma_z + \frac{\hbar \Omega}{2} \sigma_x,
	\label{eq:RWAHamiltonian}
\end{equation}





%----------------------------------------------------------------------------------------
%	SECTION 2: Applications and Implications
%----------------------------------------------------------------------------------------

\section{Applications and Implications}

Rabi oscillations and the associated theoretical tools, such as the RWA and density matrix formalism, have wide-ranging applications:
\begin{itemize}
	\item \textbf{Quantum Computing:} Rabi oscillations are used to implement quantum gates by precisely controlling the population of qubits.
	\item \textbf{Spectroscopy:} The Rabi frequency provides information about the interaction strength between light and matter.
	\item \textbf{Atomic Physics:} Understanding Rabi oscillations is essential for manipulating atomic states in experiments.
\end{itemize}

These concepts form the foundation for advanced topics in quantum mechanics and quantum technologies.


















\section*{Step-by-Step Derivation of RWA in a Two-Level System (with Phase)}
\subsection*{Step 5: Apply the Rotating Wave Approximation (RWA)}

Use the identity:
\begin{equation}
	\cos(\omega_L t + \phi) = \frac{1}{2}\left(e^{i(\omega_L t + \phi)} + e^{-i(\omega_L t + \phi)} \right)
\end{equation}

The interaction terms become:
\begin{equation}
	\frac{\hbar\Omega}{2} \left( e^{i(\omega_0 t + \omega_L t + \phi)} \ket{e}\bra{g} + e^{i(\omega_0 t - \omega_L t - \phi)} \ket{e}\bra{g} + \text{h.c.} \right)
\end{equation}

Keep only the \textbf{slowly rotating terms} at frequency \(\Delta = \omega_0 - \omega_L\), and drop the fast ones. The RWA Hamiltonian becomes:
\begin{equation}
	H_{\text{RWA}} = \hbar \Delta \ket{e}\bra{e} + \frac{\hbar \Omega}{2} \left( e^{i\phi} \ket{e}\bra{g} + e^{-i\phi} \ket{g}\bra{e} \right)
\end{equation}

\subsection*{Step 6: Solve the Dynamics (Rabi Oscillations with Phase)}

In matrix form, in the basis \(\{ \ket{e}, \ket{g} \}\):
\begin{equation}
	H_{\text{RWA}} = \frac{\hbar}{2}
	\begin{pmatrix}
		2\Delta           & \Omega e^{i\phi} \\
		\Omega e^{-i\phi} & 0
	\end{pmatrix}
\end{equation}

The phase \(\phi\) does not change the \textbf{Rabi frequency}:
\begin{equation}
	\Omega_R = \sqrt{\Delta^2 + \Omega^2}
\end{equation}
But it \textbf{rotates the axis of Rabi oscillations} in the Bloch sphere — i.e., it changes the \textbf{initial direction} of the drive.

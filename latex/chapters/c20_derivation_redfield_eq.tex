% Chapter Template
% Chapter Template

\chapter{Chapter Title Here} % Main chapter title

\label{ChapterX} % Change X to a consecutive number; for referencing this chapter elsewhere, use \ref{ChapterX}

%----------------------------------------------------------------------------------------
%	SECTION 1
%----------------------------------------------------------------------------------------

\section{Derivation of the Lindblad equation from microscopic dynamics}

The most common derivation of the Lindblad master equation is based on open quantum theory. The Lindblad equation is then an effective motion equation for a subsystem that belongs to a more complicated system. This derivation can be found in several textbooks such as Breuer and Petruccione \cite{Breuer2002} as well as Gardiner and Zoller \cite{Gardiner2004}. Here, we follow the derivation presented in Ref. \cite{Lindblad1976}. Our initial point is displayed in Fig. \ref{fig:system_env}. A total system belonging to a Hilbert space $\mathcal{H}_T$ is divided into our system of interest, belonging to a Hilbert space $\mathcal{H}_S$, and the environment living in $\mathcal{H}_E$.

The evolution of the total system is given by the von Neumann equation,
\begin{equation}
    \dot{\rho}_T(t) = -i[H_T, \rho_T(t)].
\end{equation}

As we are interested in the dynamics of the system without the environment, we trace over the environment degrees of freedom to obtain the reduced density matrix of the system $\rho(t) = \mathrm{Tr}_E[\rho_T]$. The total Hamiltonian can be separated as
\begin{equation}
    H_T = H_S \otimes \mathbb{1}_E + \mathbb{1}_S \otimes H_E + \lambda H_I,
\end{equation}
where $H_S \in \mathcal{H}_S$, $H_E \in \mathcal{H}_E$, and $H_I \in \mathcal{H}_T$ represents the interaction between the system and the environment with coupling strength $\lambda$. The interaction term is typically decomposed as
\begin{equation}
    H_I = \sum_i S_i \otimes E_i,
\end{equation}
where $S_i \in \mathcal{B}(\mathcal{H}_S)$ and $E_i \in \mathcal{B}(\mathcal{H}_E)$.

To describe the system dynamics, we move to the interaction picture where the operators evolve with respect to $H_S + H_E$,
\begin{equation}
    \tilde{O}(t) = e^{i(H_S+H_E)t} O e^{-i(H_S+H_E)t}.
\end{equation}
The time evolution in the interaction picture is given by
\begin{equation}
    \dot{\tilde{\rho}}_T(t) = -i [\tilde{H}_I(t), \tilde{\rho}_T(t)],
\end{equation}
which can be formally integrated as
\begin{equation}
    \tilde{\rho}_T(t) = \tilde{\rho}_T(0) - i \lambda \int_0^t ds [\tilde{H}_I(s), \tilde{\rho}_T(s)].
\end{equation}

Substituting this back into the evolution equation and tracing out the environment under the Born approximation ($\rho_T(t) \approx \rho_S(t) \otimes \rho_E$), we obtain the Redfield equation,
\begin{equation}
    \dot{\rho}_S(t) = -\lambda^2 \int_0^t ds \mathrm{Tr}_E [\tilde{H}_I(t), [\tilde{H}_I(s), \rho_S(t) \otimes \rho_E]].
\end{equation}

Under the Markov approximation, we extend the upper limit of integration to $\infty$, obtaining the general form of the master equation,
\begin{equation}
    \dot{\rho}_S(t) = \sum_{i,j} \Gamma_{ij} (S_j \rho_S S_i^\dagger - \frac{1}{2} \{ S_i^\dagger S_j, \rho_S \}).
\end{equation}
This is the standard form of the Lindblad master equation, where $\Gamma_{ij}$ are coefficients obtained from the spectral properties of the environment.

%-----------------------------------
%	SUBSECTION 1
%-----------------------------------
\subsection{Subsection 1}

Nunc posuere quam at lectus tristique eu ultrices augue venenatis. Vestibulum ante ipsum primis in faucibus orci luctus et ultrices posuere cubilia Curae; Aliquam erat volutpat. Vivamus sodales tortor eget quam adipiscing in vulputate ante ullamcorper. Sed eros ante, lacinia et sollicitudin et, aliquam sit amet augue. In hac habitasse platea dictumst.

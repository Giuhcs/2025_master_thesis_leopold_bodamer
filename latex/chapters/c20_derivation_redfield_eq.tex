% Chapter Template
% Chapter Template

\chapter{Chapter Title Here} % Main chapter title

\label{ChapterX} % Change X to a consecutive number; for referencing this chapter elsewhere, use \ref{ChapterX}

%----------------------------------------------------------------------------------------
%	SECTION 1
%----------------------------------------------------------------------------------------

\section{Derivation of the Lindblad equation from microscopic dynamics}

The most common derivation of the Lindblad master equation is based on open quantum theory. The Lindblad equation is then an effective motion equation for a subsystem that belongs to a more complicated system. This derivation can be found in several textbooks such as Breuer and Petruccione \cite{breuer_theory_2009} 
A total system belonging to a Hilbert space $\mathcal{H}_T$ is divided into our system of interest, belonging to a Hilbert space $\mathcal{H}_S$, and the environment living in $\mathcal{H}_E$.

The evolution of the total system is given by the von Neumann equation,
\begin{equation}
    \dot{\rho}_T(t) = -i[H_T, \rho_T(t)].
\end{equation}

As we are interested in the dynamics of the system without the environment, we trace over the environment degrees of freedom to obtain the reduced density matrix of the system $\rho(t) = \mathrm{Tr}_E[\rho_T]$. The total Hamiltonian can be separated as

\paragraph{Step 1: Interaction picture}

\begin{equation}
    H_T = H_S \otimes \mathbb{1}_E + \mathbb{1}_S \otimes H_E + \alpha H_I,
\end{equation}
where $H_S \in \mathcal{H}_S$, $H_E \in \mathcal{H}_E$, and $H_I \in \mathcal{H}_T$ represents the interaction between the system and the environment with coupling strength $\alpha$. The interaction term is typically decomposed as
\begin{equation}
    H_I = \sum_i S_i \otimes E_i,
\end{equation}
where $S_i \in \mathcal{B}(\mathcal{H}_S)$ and $E_i \in \mathcal{B}(\mathcal{H}_E)$.

To describe the system dynamics, we move to the interaction picture where the operators evolve with respect to $H_S + H_E$,
\begin{equation}
    \hat{O}(t) = e^{i(H_S+H_E)t} O e^{-i(H_S+H_E)t}.
\end{equation}

The time evolution in the interaction picture is given by
\begin{equation}
    \dot{\hat{\rho}}_T(t) = -i \alpha [\hat{H}_I(t), \hat{\rho}_T(t)],
\end{equation}

\paragraph{Step 2: Expand}
which can be formally integrated as

\begin{equation}
    \hat{\rho}_T(t) = \hat{\rho}_T(0) - i \alpha \int_0^t ds [\hat{H}_I(s), \hat{\rho}_T(s)].
\end{equation}

\begin{equation}
    \dot{\hat{\rho}}_T(t) = -i \alpha \left[ \hat{H}_I(t), \hat{\rho}_T(0) \right] - \alpha^2 \int_0^t \left[ \hat{H}_I(t), \left[ \hat{H}_I(t'), \hat{\rho}_T(t') \right] \right] dt'.
\end{equation}

\begin{equation}
    \int_t^{t'} d s \hat{\rho}_T(s) = -i \int_t^{t'} \left[ \hat{H}_I(s), \hat{\rho}_T(s) \right] ds,
\end{equation}

\begin{equation}
    \hat{\rho}_T(t') - \hat{\rho}_T(t) = -i \alpha \int_t^{t'} \left[ \hat{H}_I(s), \hat{\rho}_T(s) \right] ds.
\end{equation}

\begin{equation}
    \dot{\hat{\rho}}_T(t) = -i \alpha \left[ \hat{H}_I(t), \hat{\rho}_T(0) \right] 
    - \alpha^2 \int_0^t \left[ \hat{H}_I(t), \left[ \hat{H}_I(t'), \hat{\rho}_T(t) \right] \right] + \mathcal{O} (\alpha^3).
\end{equation}

\paragraph{Step 3: Partial trace}

\begin{equation}
    \dot{\rho}_S(t) = -i \alpha [\hat{H}_I(t),\hat{\rho}_T(0)]  - \alpha^2 \int_0^t ds \mathrm{Tr}_E [\hat{H}_I(t), [\hat{H}_I(s), \rho_S(t) \otimes \rho_E]].
\end{equation}

Approximation
\begin{equation}
    \hat{\rho}_T(0) = \hat{\rho}_S(0) \otimes \hat{\rho}_E(0)
\end{equation}
using eq. 

\begin{equation}
    \sum_i \mathrm{Tr}_E[ S_i \otimes E_i , \hat{\rho}_S(0) \otimes \hat{\rho}_E(0)] =     \sum_i (S_i \hat{\rho}_S(0) - \hat{\rho}_S(0) S_i)\cdot  \mathrm{Tr}_E [E_i \hat{\rho}_E(0)]
\end{equation}

Argue that 
\begin{equation}
    < E_i >_E = \mathrm{Tr}_E [E_i \hat{\rho}_E(0)]
\end{equation}
with new Hamiltonian
...
\begin{equation}
    \hat{H}_S' = \hat{H}_S + \sum_i S_i \otimes (E_i - < E_i >_E)
\end{equation}


TODO FINISH this derivation
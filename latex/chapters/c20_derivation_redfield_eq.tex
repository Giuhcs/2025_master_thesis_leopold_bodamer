% Chapter Template

\chapter{Derivation of the Redfield Equation} % Main chapter title

\label{Chapter:Derivation_Redfield_Equation} % Label for referencing this chapter

The following derivation is part of \cite{manzanoShortIntroductionLindblad2020}
%----------------------------------------------------------------------------------------
%	SECTION 1
%----------------------------------------------------------------------------------------

\section{Derivation from microscopic dynamics}
\label{sec:Derivation_redfield_eq_from_microscopic_dynamics}

The most common derivation of the Redfield master equation is based on open quantum theory.
The Redfield equation is then an effective motion equation for a subsystem that belongs to a more complicated system.
This derivation can be found in several textbooks such as Breuer and Petruccione \cite{breuerTheoryOpenQuantum2009}.
A total system belonging to a Hilbert space $\mathcal{H}_T$ is divided into our system of interest, belonging to a Hilbert space $\mathcal{H}_S$, and the environment living in $\mathcal{H}_E$.

The evolution of the total system is given by the von Neumann equation:

\begin{equation}
	\dot{\rho}_T(t) = -i[H_T, \rho_T(t)],
	\label{eq:Von_Neumann_Equation}
\end{equation}
where $\rho_T(t)$ is the density matrix of the total system, and $H_T$ is the total Hamiltonian.

As we are interested in the dynamics of the system without the environment, we trace over the environment degrees of freedom to obtain the reduced density matrix of the system $\rho(t) = \mathrm{Tr}_E[\rho_T]$.
The total Hamiltonian can be separated as:

\begin{equation}
	H_T = H_S \otimes \mathbb{1}_E + \mathbb{1}_S \otimes H_E + \alpha H_I,
	\label{eq:Total_Hamiltonian}
\end{equation}
where $H_S \in \mathcal{H}_S$, $H_E \in \mathcal{H}_E$, and $H_I \in \mathcal{H}_T$. $ H_I $ represents the interaction between the system and the environment with coupling strength $\alpha$.
The interaction term is typically decomposed as:

\begin{equation}
	H_I = \sum_i S_i \otimes E_i,
	\label{eq:Interaction_Hamiltonian}
\end{equation}
where $S_i \in \mathcal{B}(\mathcal{H}_S)$ and $E_i \in \mathcal{B}(\mathcal{H}_E)$ are operators, that only act on the system and environment respectively.

The following requirements are to be fuildilled by the final derived Redfield equation:

\begin{itemize}
	\item The equation should be linear in the system density matrix $ \dot{ \rho}_S(t) = F(\rho_S(t))$ (reduced equation of motion).
	\item The equation should be Markovian, meaning that the evolution of the system density matrix at time $t$ only depends on the state of the system at time $t$ and not on its past history $ \dot{ \rho}_S(t) = \rho_S(t)$ .
	\item The equation should be trace-preserving, meaning that $\mathrm{Tr}[\rho_S(t)] = \mathrm{Tr}[\rho_S(0)]$ for all times $t$.
\end{itemize}
Unlike the Linblad equation it does not guarantee the complete (not even normal) positivity of the density matrix, which is a requirement for a physical state.
So care has to be taken, when the Redfieldequation is useful.
The equation will be valid in the weak coupling limit, meaning that the constant in the interaction Hamiltonian $H_I$ fulfills  $\alpha \ll  1$.

%----------------------------------------------------------------------------------------
%	SUBSECTION 1
%----------------------------------------------------------------------------------------

\subsection{Interaction Picture}
\label{subsec:Interaction_Picture}

To describe the system dynamics, we move to the interaction picture where the operators evolve with respect to $H_S + H_E$.
And arbitrary operator $O$ in the Schroedinger picture takes the form

\begin{equation}
	\hat{O}(t) = e^{i(H_S+H_E)t} O e^{-i(H_S+H_E)t},
	\label{eq:Interaction_Picture_Operators}
\end{equation}
in the interaction picture. States evolve according to the interaction Hamiltonian $H_I$:
The time evolution of the density matrix is given in the interaction picture by the Liouville-von Neumann equation:

\begin{equation}
	\dot{\hat{\rho}}_T(t) = -i \alpha [\hat{H}_I(t), \hat{\rho}_T(t)],
	\label{eq:LiouvilleVN}
\end{equation}
which can be formally integrated as:

\begin{equation}
	\hat{\rho}_T(t) = \hat{\rho}_T(0) - i \alpha \int_0^t ds [\hat{H}_I(s), \hat{\rho}_T(s)].
	\label{eq:Formal_Integration}
\end{equation}
This equation will be inserted in Eq. \eqref{eq:LiouvilleVN}:

\begin{equation}
	\dot{\hat{\rho}}_T(t) = -i \alpha \left[ \hat{H}_I(t), \hat{\rho}_T(0) \right]
	- \alpha^2 \int_0^t \left[ \hat{H}_I(t), \left[ \hat{H}_I(s), \hat{\rho}_T(s) \right] \right] ds.
	\label{eq:Second_Order_Expansion}
\end{equation}
which is not Morkovian, because of the intergal which sums up all the past.
The step of integration and insertion can be repreated leading to a series expansion of the density matrix in the small parameter $\alpha$:

\begin{equation}
	\dot{\hat{\rho}}_T(t) = -i \alpha \underbrace{\left[ \hat{H}_I(t), \hat{\rho}_T(0) \right]}_{\text{(1)}}
	- \alpha^2 \int_0^t \underbrace{\left[ \hat{H}_I(t), \left[ \hat{H}_I(t'), \hat{\rho}_T(t') \right] \right]}_{\text{(2)}} dt' + \mathcal{O} (\alpha^3).
	\label{eq:Second_Order_Expansion_wo_third}
\end{equation}
where third order contributions (and higher) are neglected from now on.
This can be justified in the weak coupling limit where $ \alpha \ll 1 $, which represents our first approximation.
Remark, that Eq. \eqref{eq:Second_Order_Expansion_wo_third} is still not Morkovian, because of the intergal over time.

Since we are only interested in the dynamics of the system $ S $, we will now take the partial trace over the environment degrees of freedom in Eq. \eqref{eq:Second_Order_Expansion_wo_third}.

%----------------------------------------------------------------------------------------
%	SUBSECTION 2
%----------------------------------------------------------------------------------------

\subsection{Partial Trace and Approximation}
\label{subsec:Partial_Trace_Approximation}

We now assume the whole system to be seperatable at $ t = 0 $ as a product state:

\begin{equation}
	\hat{\rho}_T(0) = \hat{\rho}_S(0) \otimes \hat{\rho}_E(0),
	\label{eq:Initial_Product_State}
\end{equation}
which means, that the two sub-systems only come into contact at $ t = 0 $ and there are no correlations.
We take the interaction Hamiltonian Eq. \eqref{eq:Interaction_Hamiltonian} into account.
With this, the partial trace over the environment of the part (1) of Eq. \eqref{eq:Second_Order_Expansion_wo_third} is given by:

\begin{align}
	\sum_i \mathrm{Tr}_E\big[ S_i \otimes E_i , \hat{\rho}_S(0) \otimes \hat{\rho}_E(0)\big]
	= \sum_i \big(S_i \hat{\rho}_S(0) - \hat{\rho}_S(0) S_i\big) \cdot \mathrm{Tr}_E \big[E_i \hat{\rho}_E(0)\big],
	\label{eq:Trace_Relation_first_part}
\end{align}

where we used the cyclic property of the trace.
We define the average of the bath degrees of freedom at zero temperature:

\begin{equation}
	\langle E_i \rangle_0 \equiv \mathrm{Tr}_E \big[E_i \hat{\rho}_E(0)\big].
	\label{eq:Environment_Expectation_Value}
\end{equation}
which results to zero, simplifying Eq. \eqref{eq:Second_Order_Expansion_wo_third} to only the second part.
This can always be justified when adding a zero to the total Hamiltonian:

\begin{equation}
	H_T = H_S' + H_E + \alpha H_I',
	\label{eq:Shifted_Total_Hamiltonian}
\end{equation}
where a new interaction and system Hamiltonian are introduced.
The interaction Hamiltonian takes new environmental operators $E_i'$ which are shifted by the average of the environment operators at time $t = 0$:

\begin{equation}
	H_I' = \sum_i S_i \otimes E_i' = \sum_i S_i \otimes (E_i - \langle E_i \rangle_0).
	\label{eq:Shifted_Interaction_Hamiltonian}
\end{equation}
The new system Hamiltonian is then given by the sum of the original system Hamiltonian shifted proportionally by the average of the environment operators at time $t = 0$:

\begin{equation}
	H_S' = H_S + \alpha \sum_i S_i \langle E_i \rangle_0,
	\label{eq:Shifted_System_Hamiltonian}
\end{equation}
This however doesn't change the structure of the system dynamics.
It only accounts for a redefinition of the energy levels ("a sort of renormalization").
This way the equation \eqref{eq:Trace_Relation_first_part} results to zero and only the second part of the equation Eq. \eqref{eq:Second_Order_Expansion_wo_third} remains.

\begin{align}
	\dot{\rho}_S(t) = -i \alpha [\hat{H}_I(t),\hat{\rho}_T(0)]
	  & - \alpha^2 \int_0^t ds \mathrm{Tr}_E \big[\hat{H}_I(t), [\hat{H}_I(s), \rho_S(t) \otimes \rho_E]\big] \notag \\
	= & - \alpha^2 \int_0^t ds \mathrm{Tr}_E \big[\hat{H}_I(t), [\hat{H}_I(s), \rho_S(t) \otimes \rho_E]\big].
	\label{eq:Partial_Trace_Derivation}
\end{align}
In the follwoing, we will derive the final expression by calculating the environmental traces in the last equation.

%----------------------------------------------------------------------------------------
%	SUBSECTION 3
%----------------------------------------------------------------------------------------

\subsection{Final Expression}
\label{subsec:Final_Expression}

Defining $s' = t - s$, we rewrite the second-order term as:
\begin{align}
	\dot{\rho}_S(t)  = \alpha^2 \int_0^t ds \mathrm{Tr}_E \bigg\{
	\hat{H}_I(t) \big[ \hat{H}_I(t-s) \hat{\rho}_T(t) - \hat{\rho}_T(t) \hat{H}_I(t-s) \big] \notag \\
	- \big[ \hat{H}_I(t-s) \hat{\rho}_T(t) - \hat{\rho}_T(t) \hat{H}_I(t-s) \big] \hat{H}_I(t)
	\bigg\}.
	\label{eq:Second_Order_Final_Expression}
\end{align}
A seperatability at all times is now assumed:

\begin{equation}
	\hat{\rho}_T(t) = \hat{\rho}_S(t) \otimes \hat{\rho}_E(t),
	\label{eq:Reduced_Density_Matrix_Assumption}
\end{equation}
This assumtion has to be made even stronger later $\hat{\rho}_T(t) = \hat{\rho}_S(t) \otimes \hat{\rho}_E(0)$.
Expanding Eq. \eqref{eq:Second_Order_Final_Expression}, we obtain:

\begin{align}
	\dot{\hat{\rho}}_T(t) & =  \alpha^2 \int_0^t ds
	\bigg\{
	\mathrm{Tr}_E \big[ \hat{H}_I(t) \hat{H}_I(t-s) \hat{\rho}_T(t) \big] -
	\mathrm{Tr}_E \big[ \hat{H}_I(t) \hat{\rho}_T(t) \hat{H}_I(t-s) \big] - \notag \\
	                      & \qquad \qquad \qquad
	\mathrm{Tr}_E \big[ \hat{H}_I(t-s) \hat{\rho}_T(t) \hat{H}_I(t) \big] +
	\mathrm{Tr}_E \big[ \hat{\rho}_T(t) \hat{H}_I(t-s) \hat{H}_I(t) \big]
	\bigg\}.
	\label{eq:Expanded_Second_Order_Expression}
\end{align}

Now, inserting the interaction Hamiltonian by tracking the operators at time $t - s$ with $i'$ and at $t$ with $i$, we have:

\begin{align}
	\dot{\hat{\rho}}_T(t) & = \alpha^2  \sum_{i, i'} \int_0^t ds
	\bigg\{
	\mathrm{Tr}_E \big[ \hat{S}_i(t) \hat{S}_{i'}(t-s) \hat{\rho}_S(t)      \otimes   \hat{E}_{i}(t) \hat{E}_{i'}(t-s) \hat{\rho}_E(t)  \big] -  \notag                         \\
	                      & \mathrm{Tr}_E \big[ \hat{S}_i(t) \hat{\rho}_S(t) \hat{S}_{i'}(t-s)      \otimes   \hat{E}_{i}(t) \hat{\rho}_E(t) \hat{E}_{i'}(t-s)  \big] - \notag  \\
	                      & \mathrm{Tr}_E \big[ \hat{S}_{i'}(t-s) \hat{\rho}_S(t) \hat{S}_i(t)      \otimes   \hat{E}_{i'}(t-s) \hat{\rho}_E(t) \hat{E}_{i}(t)  \big] +  \notag \\
	                      & \mathrm{Tr}_E \big[ \hat{\rho}_S(t) \hat{S}_{i'}(t-s) \hat{S}_i(t)      \otimes   \hat{\rho}_E(t) \hat{E}_{i'}(t-s) \hat{E}_{i}(t)  \big]
	\bigg\}.
	\label{eq:Interaction_Hamiltonian_Expansion}
\end{align}

Since the trace only acts on the environment, the system operators can be taken out of the trace, and we define the correlation functions:
\begin{equation}
	C_{ij}(t - s) = \mathrm{Tr}_E \big[\hat{E}_{i}(t) \hat{E}_{i'}(t-s) \hat{\rho}_E(t)\big],
	\label{eq:Environment_Correlation_Function}
\end{equation}
such that:
\begin{align}
	\dot{\hat{\rho}}_T(t) = \alpha^2  \sum_{i, i'} \int_0^t ds
	\bigg\{
	C_{ij}(t - s) \big[ \hat{S}_i(t),  \hat{S}_{i'}(t-s) \hat{\rho}_S(t) \big] + \text{H.c.}
	\bigg\}.
	\label{eq:Redfield_Equation_Final}
\end{align}
which is the desired form of the Redfield equation.

Note however, that the we have not used the strong condition $\hat{\rho}_T(t) = \hat{\rho}_S(t) \otimes \hat{\rho}_E(0)$ was not used yet.
This however will make it possible to calculate the correlations Eq. \eqref{eq:Environment_Correlation_Function}, because we can assume that the enviroment is in a thermal equilibruium at a certain temperature.
Because of the assumption this is the case for all times.
It is equivalent to say that the environment is unaffected by the system. It is memoryless, because it is very big.
% !TEX root = ../main.tex
\chapter{Open Quantum Systems} % Main chapter title
\label{Chapter:Open_Quantum_Systems} % Label for referencing this chapter

%------------------------------------------------------------------------------
%	SECTION 1: Introduction to Open Quantum Systems
%------------------------------------------------------------------------------

\section{Introduction to Open Quantum Systems}
\label{sec:introduction_open_quantum_systems}

In the realm of quantum mechanics, most real-world quantum systems are not perfectly isolated. Instead, they interact with their surrounding environment, leading to the concept of \emph{open quantum systems}. Unlike closed quantum systems that evolve unitarily according to the Schrödinger equation, open quantum systems experience non-unitary evolution due to their coupling with an external environment or reservoir.
Whenever the system is driven out of equilibrium by external perturbations, this coupling makes the system relax back to equilibrium.
\subsection{What are Open Quantum Systems?}
Open quantum systems are quantum systems that interact with an external environment, often called a "bath," which typically has many degrees of freedom and is usually assumed to be in thermal equilibrium~\cite{BreuerPetruccione2009TheoryOpenQuantum, Weiss2012QuantumDissipativeSystems}. This interaction leads to several fundamental phenomena that distinguish open systems from isolated, closed quantum systems.

A central effect is \textbf{decoherence}, where quantum superposition states lose their phase relationships due to entanglement with the environment (\textbf{dephasing}). As a result, pure quantum states evolve into classical statistical mixtures, and the system's ability to exhibit quantum interference is diminished. The characteristic time scale for this process, known as the \emph{coherence time}, is of utmost importance in experiments, especially in high-resolution spectroscopy, quantum information processing, and quantum optics. For example, in two-dimensional electronic spectroscopy, the coherence time determines the ability to resolve quantum beats and thus extract information about energy transfer pathways in complex molecular systems~\cite{Mukamel1995PrinciplesNonlinearOptical}.

In addition to dephasing, the system-environment interaction leads to \textbf{dissipation} or \textbf{thermalization}, where energy is exchanged between the system and its surroundings, causing the system to relax toward a thermal state determined by the environment's temperature.

Understanding and controlling these effects is crucial for the design of quantum technologies, such as quantum computers and sensors, where environmental noise can rapidly destroy fragile quantum states~\cite{Schlosshauer2007DecoherenceBook, LaddEtAl2010QuantumComputers}.

In summary, open quantum systems theory provides the framework to describe how quantum systems lose their "quantumness" and transition toward classical behavior due to unavoidable interactions with their environment.

\subsection{Origins and Physical Motivation}

Open quantum systems theory emerges naturally from the realization that perfect isolation of quantum systems is practically impossible; every quantum system in nature interacts with its environment to some degree. For instance, atoms are subject to electromagnetic field fluctuations (vacuum fluctuations)~\cite{BreuerPetruccione2009TheoryOpenQuantum}, quantum dots in solid-state environments couple to phonon baths~\cite{Weiss2012QuantumDissipativeSystems}, and molecular systems interact with surrounding solvent molecules~\cite{Mukamel1995PrinciplesNonlinearOptical}. Quantum computers are particularly sensitive to environmental noise, which can rapidly destroy quantum coherence~\cite{LaddEtAl2010QuantumComputers}, while biological quantum systems such as photosynthetic complexes operate in inherently noisy cellular environments~\cite{Schlosshauer2007DecoherenceBook}. These diverse scenarios all require a theoretical framework that accounts for the coexistence of quantum effects and environmental influences.


\subsection{Theoretical Approaches}

A variety of theoretical frameworks have been developed to describe the dynamics of open quantum systems, each with its own range of validity and underlying assumptions. The most important distinction among these approaches is whether they assume \textbf{Markovian} or \textbf{non-Markovian} dynamics.

\subsubsection{Markovian vs. Non-Markovian Dynamics}
\label{subsec:markovian_nonmarkovian}

\noindent
\textbf{Markovian dynamics} assume that the environment has no memory: the future evolution of the system depends only on its present state, not on its past history. This is valid when the environmental correlation time is much shorter than the system's characteristic timescale. In contrast, \textbf{non-Markovian dynamics} account for memory effects, where the environment retains information about the system's past, leading to feedback and more complex evolution~\cite{BreuerPetruccione2009TheoryOpenQuantum, Rivas2014QuantumNonMarkovianityReview}.

\subsubsection{Overview of Main Approaches}

\paragraph{Master Equation Approaches}
\begin{itemize}
    \item \textbf{Lindblad Master Equation} (Markovian): Provides a general form for completely positive, trace-preserving (CPTP) dynamics under the Markovian approximation. Widely used in quantum optics and quantum information~\cite{BreuerPetruccione2009TheoryOpenQuantum, Lindblad1976GeneratorsQuantumDynamical}.
    \item \textbf{Redfield Equation} (Markovian, but not always CPTP): Derived from a microscopic Hamiltonian using the Born-Markov approximation. Captures both dissipation and dephasing, but does not guarantee complete positivity~\cite{Redfield1965TheoryRelaxationProcesses,
              RivasEtAl2014QuantumNonmarkovianityCharacterization,
              LiEtAl2018ConceptsQuantumNonmarkovianity}.
\end{itemize}

\paragraph{Non-Markovian Approaches}
\begin{itemize}
    \item \textbf{Time-Convolutionless (TCL) and Nakajima-Zwanzig Equations}: Generalized master equations that include memory kernels, allowing for non-Markovian effects~\cite{BreuerPetruccione2009TheoryOpenQuantum, Rivas2014QuantumNonMarkovianityReview}.
    \item \textbf{Hierarchical Equations of Motion (HEOM)}: Numerically exact approach for strongly coupled and non-Markovian environments, especially in condensed phase systems~\cite{Tanimura2020HEOMReview}.
    \item \textbf{Path Integral Methods}: The Feynman-Vernon influence functional and related techniques provide a non-perturbative, non-Markovian description of system-bath dynamics~\cite{Weiss2012QuantumDissipativeSystems}.
\end{itemize}

\paragraph{Stochastic Approaches}
\begin{itemize}
    \item \textbf{Stochastic Schrödinger Equations and Quantum Trajectories}: These methods unravel the master equation into individual realizations, providing insight into measurement backaction and quantum jumps~\cite{BreuerPetruccione2009TheoryOpenQuantum, Carmichael1993OpenSystemsBook}.
\end{itemize}


\subsection{The Redfield Equation: A Central Tool}

Among the various master equation approaches to open quantum systems, the Redfield equation occupies a particularly important position due to its microscopic foundation and broad applicability~\cite{Redfield1965TheoryRelaxationProcesses, BreuerPetruccione2009TheoryOpenQuantum}. Originally developed by A.G. Redfield in 1957 and 1965 for nuclear magnetic resonance relaxation phenomena~\cite{Redfield1965TheoryRelaxationProcesses}, the Redfield equation describes the time evolution of the reduced density matrix of a quantum system weakly coupled to a thermal environment.

The Redfield master equation provides a systematic perturbative treatment of system-environment interactions within the Born-Markov approximation. Unlike phenomenological approaches that introduce dissipation and dephasing \emph{ad hoc}, the Redfield formalism derives these effects directly from the microscopic Hamiltonian of the total system, making it particularly valuable for understanding the fundamental mechanisms underlying quantum decoherence~\cite{BreuerPetruccione2009TheoryOpenQuantum, Weiss2012QuantumDissipativeSystems}.

The key advantages of the Redfield approach include:
\begin{itemize}
    \item \textbf{Microscopic foundation}: Derived rigorously from the von Neumann equation for the total system-environment composite, ensuring consistency with fundamental quantum mechanics~\cite{BreuerPetruccione2009TheoryOpenQuantum}
    \item \textbf{Physical transparency}: The relaxation rates are directly related to environmental correlation functions and spectral densities, providing clear physical insight into the dissipation mechanisms~\cite{Redfield1965TheoryRelaxationProcesses}
\end{itemize}

However, it is important to note that the Redfield equation, while preserving trace and Hermiticity, does not guarantee complete positivity of the density matrix—a fundamental requirement for physical quantum states~\cite{RivasEtAl2010MarkovianMasterEquations}. This limitation restricts its validity to the weak-coupling regime and requires careful analysis of its applicability in specific physical situations.

In this chapter, we will derive the Redfield equation starting from the fundamental microscopic dynamics of the system-environment composite and demonstrate how it emerges as an effective description for the reduced system dynamics. We will then examine the environmental correlation functions and spectral densities that characterize the bath properties and determine the system's relaxation behavior.

%------------------------------------------------------------------------------
%	SECTION 2: Derivation of the Redfield Equation
%------------------------------------------------------------------------------

\section{Derivation of the Redfield Equation from Microscopic Dynamics}
\label{sec:Derivation_redfield_eq_from_microscopic_dynamics}

The derivation of the Redfield master equation is based on open quantum theory and provides a systematic way to obtain an effective equation of motion for a quantum system coupled to an environment. This derivation follows the approach presented in \cite{Manzano2020ShortIntroductionLindblad} and can also be found in standard textbooks such as Breuer and Petruccione \cite{BreuerPetruccione2009TheoryOpenQuantum}.

\subsection{Setup: System + Environment}

We consider a quantum system of interest interacting with a larger environment. The total system Hilbert space $\mathcal{H}_T$ is the tensor product of the system Hilbert space $\mathcal{H}_S$ and the environment Hilbert space $\mathcal{H}_E$:

\begin{equation}
    \mathcal{H}_T = \mathcal{H}_S \otimes \mathcal{H}_E.
    \label{eq:Total_Hilbert_Space}
\end{equation}

The evolution of the total system is governed by the Liouville-Von Neumann equation:

\begin{equation}
    \dot{\rho}_T(t) = -i[H_T, \rho_T(t)],
    \label{eq:Von_Neumann_Equation}
\end{equation}

where $\rho_T(t)$ is the density matrix of the total system, $H_T$ is the total Hamiltonian, and we use units where $\hbar = 1$.

The total Hamiltonian can be decomposed as:

\begin{equation}
    H_T = H_S \otimes \mathbb{1}_E + \mathbb{1}_S \otimes H_E + \alpha H_I,
    \label{eq:Total_Hamiltonian}
\end{equation}

where:
\begin{itemize}
    \item $H_S$ is the system Hamiltonian
    \item $H_E$ is the environment (bath) Hamiltonian
    \item $H_I$ represents the interaction between system and environment
    \item $\alpha$ is the coupling strength parameter
\end{itemize}

The interaction Hamiltonian is typically written in the form:

\begin{equation}
    H_I = \sum_i S_i \otimes E_i,
    \label{eq:Interaction_Hamiltonian}
\end{equation}

where $S_i$ are system operators and $E_i$ are environment operators.

\subsection{Reduced Dynamics and Approximations}

Since we are interested in the dynamics of the system alone, we define the reduced density matrix by tracing over the environmental degrees of freedom:

\begin{equation}
    \rho(t) = \mathrm{Tr}_E[\rho_T(t)].
    \label{eq:Reduced_Density_Matrix}
\end{equation}

The exact equation of motion for $\rho(t)$ involves the full many-body dynamics of the environment, which is generally intractable. The Redfield approach provides a systematic approximation scheme by making several key assumptions:

\begin{enumerate}
    \item \textbf{Weak coupling approximation}: The system-environment coupling is treated perturbatively (small $\alpha$)
    \item \textbf{Markovian approximation}: The environment correlation time is much shorter than the system evolution time scale
    \item \textbf{Born approximation}: The total state remains close to a product state $\rho_S(t) \otimes \rho_E$
\end{enumerate}

The following requirements must be fulfilled by the final derived Redfield equation:

\begin{itemize}
    \item The equation should be linear in the system density matrix $\dot{\rho}_S(t) = F(\rho_S(t))$ (reduced equation of motion).
    \item The equation should be Markovian, meaning that the evolution of the system density matrix at time $t$ only depends on the state of the system at time $t$ and not on its past history.
    \item The equation should be trace-preserving, meaning that $\mathrm{Tr}[\rho_S(t)] = \mathrm{Tr}[\rho_S(0)]$ for all times $t$.
\end{itemize}

Unlike the Lindblad equation, the Redfield equation does not guarantee the complete positivity of the density matrix, which is a requirement for a physical state. Care must be taken when determining when the Redfield equation is useful. The equation will be valid in the weak coupling limit, meaning that the coupling strength in the interaction Hamiltonian fulfills $\alpha \ll 1$.

\subsection{Interaction Picture}

To describe the system dynamics, we move to the interaction picture where the operators evolve with respect to $H_S + H_E$. Any arbitrary operator $O$ in the Schrödinger picture takes the form:

\begin{equation}
    \hat{O}(t) = e^{i(H_S+H_E)t} O e^{-i(H_S+H_E)t},
    \label{eq:Interaction_Picture_Operators}
\end{equation}

in the interaction picture. States now evolve only according to the interaction Hamiltonian $H_I$, and the Liouville-Von Neumann equation becomes:

\begin{equation}
    \dot{\hat{\rho}}_T(t) = -i \alpha [\hat{H}_I(t), \hat{\rho}_T(t)],
    \label{eq:LiouvilleVN}
\end{equation}

which can be formally integrated as:

\begin{equation}
    \hat{\rho}_T(t) = \hat{\rho}_T(0) - i \alpha \int_0^t ds [\hat{H}_I(s), \hat{\rho}_T(s)].
    \label{eq:Formal_Integration}
\end{equation}

Inserting this back into the equation of motion yields:

\begin{equation}
    \dot{\hat{\rho}}_T(t) = -i \alpha \left[ \hat{H}_I(t), \hat{\rho}_T(0) \right]
    - \alpha^2 \int_0^t \left[ \hat{H}_I(t), \left[ \hat{H}_I(s), \hat{\rho}_T(s) \right] \right] ds,
    \label{eq:Second_Order_Expansion}
\end{equation}

The iteration can be repeated, leading to a series expansion in powers of $\alpha$:

\begin{equation}
    \dot{\hat{\rho}}_T(t) = -i \alpha \underbrace{\left[ \hat{H}_I(t), \hat{\rho}_T(0) \right]}_{\text{(1)}}
    - \alpha^2 \int_0^t \underbrace{\left[ \hat{H}_I(t), \left[ \hat{H}_I(s), \hat{\rho}_T(s) \right] \right]}_{\text{(2)}} ds + \mathcal{O} (\alpha^3).
    \label{eq:Second_Order_Expansion_truncated}
\end{equation}

We truncate at second order, justified by the weak coupling assumption ($\alpha \ll 1$), which represents the \textbf{Born approximation}.

\subsection{Partial Trace and Markovian Approximation}

We assume the total system starts in a product state:

\begin{equation}
    \hat{\rho}_T(0) = \hat{\rho}_S(0) \otimes \hat{\rho}_E(0),
    \label{eq:Initial_Product_State}
\end{equation}

Taking the partial trace over the environment, the first-order term vanishes if we assume $\langle E_i \rangle_0 = \mathrm{Tr}_E[E_i \hat{\rho}_E(0)] = 0$. This can always be achieved by shifting the interaction operators appropriately.

The reduced equation of motion becomes:

\begin{equation}
    \dot{\rho}_S(t) = - \alpha^2 \int_0^t ds \, \mathrm{Tr}_E \big[\hat{H}_I(t), [\hat{H}_I(s), \hat{\rho}_S(t) \otimes \hat{\rho}_E(0)]\big].
    \label{eq:Partial_Trace_Derivation}
\end{equation}

\subsection{Final Redfield Equation}

Substituting the interaction Hamiltonian and performing the environmental trace, we obtain:

\begin{equation}
    \dot{\rho}_S(t) = \alpha^2 \sum_{i,j} \int_0^t ds \, C_{ij}(t-s) \left[ \hat{S}_i(t), \hat{S}_j(t-s) \hat{\rho}_S(t) \right] + \text{H.c.},
    \label{eq:Redfield_Equation_Final}
\end{equation}

where the environmental correlation functions are defined as:

\begin{equation}
    C_{ij}(\tau) = \mathrm{Tr}_E[\hat{E}_i(\tau) \hat{E}_j(0) \hat{\rho}_E(0)].
    \label{eq:Environment_Correlation_Function}
\end{equation}

This is the Redfield master equation, describing the reduced dynamics of the system under the influence of environmental correlations.

%------------------------------------------------------------------------------
%	SECTION 3: Environmental Correlation Functions and Spectral Properties  
%------------------------------------------------------------------------------

\section{Environmental Correlation Functions and Spectral Properties}
\label{sec:environmental_correlation_functions}

Having derived the Redfield equation, we now turn our attention to one of its most crucial ingredients: the characterization of the environmental properties through correlation functions and spectral densities. These quantities encode all the relevant information about how the environment affects the system dynamics and are essential for any practical application of the Redfield formalism.

The transition from the abstract mathematical framework of the Redfield equation to concrete, calculable expressions requires a detailed understanding of the bath correlation functions. These functions not only determine the strength and time scales of the system-environment interaction but also reveal the fundamental connection between quantum and classical descriptions of environmental noise.

In the following sections, we will explore how environmental correlation functions arise naturally in the Redfield framework, examine their physical interpretation, and discuss their relationship to experimentally accessible quantities such as power spectra and spectral densities. This analysis will provide the foundation for understanding how different types of environments—from thermal baths to structured reservoirs—influence quantum system dynamics.

\subsection{Bath Correlation Functions}

The environmental correlation functions that appear in the Redfield equation are defined as:

\begin{equation}
    C_{ij}(\tau) = \langle E_i(\tau) E_j(0) \rangle_E,
    \label{eq:bath_correlator_general}
\end{equation}

where $E_i(\tau)$ represents the environment operators in the interaction picture, and $\langle \cdot \rangle_E$ denotes the expectation value with respect to the environmental state.

\subsection{Physical Interpretation: Emission and Absorption Processes}

A fundamental question when applying the Redfield formalism concerns whether both emission and absorption processes are correctly included in the theoretical description. This is particularly important when studying spectroscopic phenomena, where the competition between these processes determines the observed signals and the approach to thermal equilibrium.

The answer is \textbf{yes}: both spontaneous and stimulated emission, as well as thermal and induced absorption, are naturally included through the proper choice of the spectral function $S(\omega)$ that characterizes the system-environment interaction, provided this function satisfies the Kubo-Martin-Schwinger (KMS) condition for detailed balance.

The KMS condition requires that the spectral function satisfies:

\begin{equation}
    S(-\omega) = e^{-\hbar\omega/(k_B T)} S(\omega),
    \label{eq:kms_condition}
\end{equation}

where $T$ is the temperature of the thermal environment. This condition ensures:

\begin{itemize}
    \item \textbf{Emission processes}: Transitions with $\omega > 0$ correspond to energy transfer from system to environment
    \item \textbf{Absorption processes}: Transitions with $\omega < 0$ correspond to energy transfer from environment to system
    \item \textbf{Detailed balance}: The ratio of forward and reverse transition rates follows the Boltzmann distribution
\end{itemize}

%------------------------------------------------------------------------------
%	SECTION 3: Environmental Correlation Functions and Spectral Properties  
%------------------------------------------------------------------------------

\section{Environmental Correlation Functions and Spectral Properties}
\label{sec:environmental_correlation_functions}

Having derived the Redfield equation, we now turn our attention to one of its most crucial ingredients: the characterization of the environmental properties through correlation functions and spectral densities. These quantities encode all the relevant information about how the environment affects the system dynamics and are essential for any practical application of the Redfield formalism.

The transition from the abstract mathematical framework of the Redfield equation to concrete, calculable expressions requires a detailed understanding of the bath correlation functions. These functions not only determine the strength and time scales of the system-environment interaction but also reveal the fundamental connection between quantum and classical descriptions of environmental noise.

In what follows, we will explore how environmental correlation functions arise naturally in the Redfield framework, examine their physical interpretation, and discuss their relationship to experimentally accessible quantities such as power spectra and spectral densities. This analysis provides the foundation for understanding how different types of environments—from thermal baths to structured reservoirs—influence quantum system dynamics.

\subsection{Definition and Properties of Bath Correlation Functions}

The environmental correlation functions that appear in the Redfield equation, as defined in Eq.~\eqref{eq:Environment_Correlation_Function}, capture the temporal correlations in the environmental fluctuations:

\begin{equation}
    C_{ij}(\tau) = \mathrm{Tr}_E[\hat{E}_i(\tau) \hat{E}_j(0) \hat{\rho}_E(0)].
    \label{eq:bath_correlator_detailed}
\end{equation}

These correlation functions possess several important properties:

\begin{itemize}
    \item \textbf{Stationarity}: For environments in thermal equilibrium, $C_{ij}(\tau)$ depends only on the time difference $\tau$, not on the absolute time
    \item \textbf{Hermiticity}: $C_{ij}^*(\tau) = C_{ji}(-\tau)$ due to the Hermitian nature of the environment operators
    \item \textbf{Decay behavior}: The correlation functions typically decay on time scales characteristic of the environmental dynamics
\end{itemize}

The spectral density, obtained through Fourier transform of the correlation function, provides complementary information about the frequency content of the environmental fluctuations:

\begin{equation}
    S_{ij}(\omega) = \int_{-\infty}^{\infty} d\tau \, e^{i\omega\tau} C_{ij}(\tau).
    \label{eq:spectral_density}
\end{equation}

This spectral density directly enters the Redfield rates and determines the frequency-dependent coupling between system and environment.

% !TEX root = ../main.tex
\chapter{Introduction} % Main chapter title
\label{Chapter_Introduction} % Change X to a consecutive number; for referencing this chapter elsewhere, use \ref{ChapterX}

%-------------------------------------------------------------------------------
%	SECTION 1: Coherence and Excitation Transport
%-------------------------------------------------------------------------------

\section{Coherence and Excitation Transport}

In this chapter, we aim to explain the phenomena of long coherences (lifetimes) and the excitation transport of light on a microtubule. The proposed model takes the following approach:

\begin{itemize}
	\item The microtubule is modeled as a cylindrical structure consisting of nodes. Each node represents an atom, which is modeled as a two-level system. The number of atoms, \( N_{\text{atoms}} \), is determined by the number of chains (\( n_{\text{chains}} \)) and the number of rings (\( n_{\text{rings}} \)), assuming fixed positions for these nodes.
	\item The system is restricted to a single excitation.
	\item A time-dependent coupling to an electric field is proposed, which may be either classical or quantum in nature. This coupling is intended to facilitate spectroscopy.
	\item Two types of Lindblad operators are introduced to model dissipation processes. Specifically:
	      \begin{enumerate}
		      \item Spontaneous decay
		      \item Dephasing
	      \end{enumerate}
\end{itemize}
The Lindblad operators introduced to model the spontaneous decay and dephasing processes for each individual atom are defined as follows:

\begin{align}
	C_{\text{decay}}^{(i)}   & = \sqrt{\gamma_0} \, \sigma_-^{(i)},    \\
	C_{\text{dephase}}^{(i)} & = \sqrt{\gamma_\phi} \, \sigma_z^{(i)},
\end{align}

where:
\begin{itemize}
	\item \( C_{\text{decay}}^{(i)} \) describes the spontaneous decay of the \(i\)-th atom, with a rate given by \(\gamma_0\).
	\item \( C_{\text{dephase}}^{(i)} \) describes the dephasing of the \(i\)-th atom, with a rate given by \(\gamma_\phi\).
	\item \( \sigma_-^{(i)} \) is the lowering operator for the \(i\)-th atom, and \( \sigma_z^{(i)} \) is the Pauli \( z \)-operator for the \(i\)-th atom.
\end{itemize}


\todoidea{Give a whole introduction to quantum biology, why it is interesting, quantum consciousness, ... , microtubules, coherence, excitation transport, 2D spectroscopy, with open quantum systems}
\newpage











\section{Motivation}
\noindent

%----------------------------------------------------------------------------------------
%	SECTION 1
%----------------------------------------------------------------------------------------
%\section{Objective}
%\vspace{0.5cm}
%\noindent
%The goal of this thesis is
%to perform robust directional photon routing on atomic systems in free-space using subradiant states.
%Focusing on a Y-shaped atomic tree, different topologies are explored to enable long-lived information transport as a proof of concept.
%
%\section{Outline}
%This thesis is structured as follows.
%Chapter \ref{Chapter2} introduces the theoretical background.
%It covers the concepts of open quantum systems,
%subradiance and superradiance, the Green tensor, and the reciprocal space.
%These tools are essential foundations for describing atom-atom interactions in free space,
%including dipole-dipole interactions and coupling to a photonic bath.
%%After this chapter, the reader already knows...
%The quantum router of \cite{startingpoint} is presented and summarized in Chapter \ref{Chapter3}.
%It introduces the concepts of graph theory and explains how quantum evolution on a graph topology can be utilized to achieve directional routing of information.
%Chapter \ref{Chapter4} will be the core of this thesis, adapting this model to an atomic system.
%This chapter delves into the challenges of implementing directional routing in a fully connected atomic system and investigates various solutions to control the phase of interactions.
%It further extends the analysis to systems with a larger number of atoms, focusing on coupling control and routing capabilities in different configurations, such as equilateral and isosceles triangles.
%Chapter \ref{Chapter5} concludes the thesis by summarizing the results and discussing potential future directions in the field of quantum routing in atomic systems.


It is widely assumed that one of the crucial tasks currently facing quantum theorists
is to understand and characterize the behaviour of realistic quantum systems. In
any experiment, a quantum system is subject to noise and decoherence due to the
unavoidable interaction with its surroundings. The theory of open quantum systems
aims at developing a general framework to analyze the dynamical behaviour of systems
that, as a result of their coupling with environmental degrees of freedom, will no
longer evolve unitarily. \cite{rivasetal2010markovianmasterequations}
\\
2DES> \todoref{rind ref}%\cite{krumlandetal2023twodimensionalelectronicspectroscopy}, 
\cite{segarra-martietal2018accuratesimulationtwodimensional}, \cite{sunetal2024twodimensionalspectroscopyopen}
\\
NONlinear Optics> \cite{hamm2005principlesnonlinearoptical}, \cite{mukamel1995principlesnonlinearoptical}
\\
Spectroscopy investigates the interaction between matter and electromagnetic radiation, offering a means to analyze composition and structure.
Central to this analysis is the understanding of how molecules respond to specific frequencies of light, revealing information about their energy levels and bonding.
Key concepts include wavelength ($\lambda$), wavenumber ($\bar{\nu}$), and frequency ($\nu$).
Wavelength, the distance between successive wave crests, is typically measured in nanometers or micrometers.
Wavenumber, expressed in inverse centimeters (cm$^{-1}$), represents the number of waves per unit distance and is directly proportional to energy, defined as $\bar{\nu} = 1/\lambda$ (where $\lambda$ is in cm).
Frequency, the number of wave cycles per second, is measured in Hertz (Hz), and the angular frequency ($\omega$) is related to frequency by $\omega = 2\pi\nu$.
The relationship between angular frequency and wavenumber is given by $\omega = 2\pi c \bar{\nu}$, where $c$ is the speed of light.\\
Next, I converted all units into femtoseconds ($fs^{-1}$), which is commonly used in time-domain spectroscopy.\\
Spectrometers are instruments designed to measure the intensity of light as a function of wavelength or frequency.\\
Different types of spectrometers are employed for various regions of the electromagnetic spectrum.
Notably, UV-Vis spectrometers analyze absorption and transmission of ultraviolet and visible light, while infrared (IR) spectrometers measure the absorption of infrared light, providing insights into molecular vibrations.
Nuclear Magnetic Resonance (NMR) spectrometers probe the magnetic properties of atomic nuclei, revealing molecular structure.

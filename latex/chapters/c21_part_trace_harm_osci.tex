% Chapter Template

\chapter{Bath Correlation functions} % Main chapter title

\label{Chapter_calc_extra_stuff} % Label for referencing this chapter

%----------------------------------------------------------------------------------------
%	SECTION 1
%----------------------------------------------------------------------------------------
\subsection{Bath Correlator}
\label{subsec:bath_correlator}

The general task of this chapter is to calculate the  bath correlator, which is defined as:
\begin{equation} \label{eq:bath_correlator}
C(\tau) = \langle B(\tau) B(0) \rangle,
\end{equation}
It is often also called two point correlation function and can bee seen as measuring the operator \( B \) at two different times in the bath.
The bath operator \( B \) will be specified later.

\section{Usefull tools}

\subsection{Infinite Geometric Series}
An infinite geometric series is a series of the form:
\begin{equation} \label{eq:infinite_geometric_series}
S = a + ar + ar^2 + ar^3 + \dots = \sum_{n=0}^{\infty} ar^n,
\end{equation}
where \(a\) is the first term and \(r\) is the common ratio. If \(|r| < 1\), the sum converges to:
\begin{equation} \label{eq:geometric_series_sum}
S = \frac{a}{1 - r}.
\end{equation}

Derivation of the Eq. \eqref{eq:geometric_series_sum} with respect to $ r $ returns another usefull series, to be used in the subsequent calculations:
\begin{equation} \label{eq:derivation_geometric_sum}
\sum_{n=0}^{\infty} n r^n = \frac{r}{(1-r)^2}, \quad \text{for } |r| < 1,
\end{equation}


%----------------------------------------------------------------------------------------
%	SECTION 2
%----------------------------------------------------------------------------------------

\subsection{Trace}
If we have a bipartite system \( A \otimes B \), the reduced density matrix of system \( A \) is obtained by tracing out system \( B \):
\begin{equation} \label{eq:partial_trace}
\rho_A = \Tr_B[\rho_{AB}].
\end{equation}
If one is interested in the expectation value of an operator \( A \) in a system \( S \), this can be calculated with the trace:
\begin{equation} \label{eq:expectation_value}
\langle A \rangle_S = \Tr_S[\rho_S A] = \frac{1}{Z} \sum_n e^{-\beta E_n} A_{nn},
\end{equation}
where the inverse temperature \(\beta\) is defined as:
\begin{equation} \label{eq:beta_definition}
\beta = \frac{1}{k_B T}.
\end{equation}

%----------------------------------------------------------------------------------------
%	SECTION 3
%----------------------------------------------------------------------------------------

\section{Harmonic Oscillators}
A bosonic bath is modeled by an infinitely big set of harmonic oscillators, which we assume to be in thermal equilibrium.
For such a system the thermal state is described by the Gibbs distribution:

\begin{equation} \label{eq:gibbs_state}
\rho = \frac{e^{-\beta H}}{Z}, \quad H = \sum_k \hbar \omega_k \left(a_k^{\dagger} a_k + \frac{1}{2}\right),
\end{equation}

\subsection{Single Mode}
For a single mode \( k \) harmonic oscillator \( H = \hbar \omega_k a_k^{\dagger} a_k \) in thermal equilibrium at temperature \( T \), the density matrix is given by:

\begin{equation} \label{eq:single_mode_density_matrix}
\rho = \frac{e^{-\beta \hbar \omega_k a_k^{\dagger} a_k}}{Z},
\end{equation}
where \( \omega_k \) is the constant frequency of the mode \( k \).
The partition function can be calculated by the geometric series:

\begin{equation} \label{eq:partition_function}
    Z = \Tr\left[e^{-\beta H}\right] = \sum_{m=0}^{\infty} \langle m | e^{-\beta \hbar \omega_k a_k^{\dagger} a_k} | m \rangle
    = \sum_{k=0}^{\infty} \langle m | m \rangle \delta_{km} e^{-\beta \hbar \omega_k k} \\
%    = e^{-\beta \frac{\hbar \omega_k }{2}} \sum_{k=0}^{\infty} e^{-\beta \hbar \omega_k k} 
    = \frac{1}{1 - e^{-\beta \hbar \omega_k}}. % e^{-\beta \frac{\hbar \omega_k}{2}}
\end{equation}
where \(|n\rangle\) are number states. 
Using this, the expectation value of the number operator can be calculated:

\begin{align} \label{eq:expectation_number_operator}
n_k &= \langle a_k^{\dagger} a_k \rangle_{\text{th}} = \Tr \left[ a_k^{\dagger} a_k \frac{e^{-\beta H}}{Z} \right] \\
    &= \frac{\Tr \left[ a_k^{\dagger} a_k e^{-\beta \hbar \omega_k a_k^{\dagger} a_k} \right]}{Z} \\
    &= \frac{\sum_{m=0}^{\infty} \langle m|a_k^{\dagger} a_k e^{-\beta \hbar \omega_k a_k^{\dagger} a_k}|m \rangle}{\frac{1}{1 - e^{-\beta \hbar \omega_k}}} \\
%    &= \frac{e^{-\beta \hbar \omega_k}}{1 - e^{-\beta \hbar \omega_k}} \\
    &= \frac{1}{e^{\beta \hbar \omega_k} - 1}.
\end{align}
where we have used Eq. \eqref{eq:derivation_geometric_sum} in the last step.

The partition function for the infinite set then generalizes to a product:
\begin{equation} \label{eq:generalized_partition_function}
Z = \prod_k \frac{e^{-\beta \hbar \omega_k / 2}}{1 - e^{-\beta \hbar \omega_k}}.
\end{equation}

%----------------------------------------------------------------------------------------
%	SECTION 4
%----------------------------------------------------------------------------------------

\section{Bath Correlators}
\label{sec:bath_corr_trans_rates}

Now we turn to calculating the Correlator.
The bath operator \( B \) is:
\begin{equation} \label{eq:bath_operator}
B = \sum_{n=1}^{\infty} c_n x_n.
\end{equation}
With this definition note, that in sec. \ref{sec:sec:Derivation_redfield_eq_from_microscopic_dynamics}} the interaction hamiltonian reduces, such that only one bath correlator $C(\tau) \equiv C_{ii}(\tau)$ in Eq. \eqref{eq:Environment_Correlation_Function} remains.

Expressing \( B \) in terms of creation and annihilation operators:
\begin{align}
B(0) &= \sum_{n=1}^{\infty} c_n \sqrt{\frac{1}{2 m_n \omega_n}} (b_n + b_n^\dagger), \label{eq:bath_operator_t0} \\
B(\tau) &= \sum_{n=1}^{\infty} c_n \sqrt{\frac{1}{2 m_n \omega_n}} \left( b_n e^{-i \omega_n \tau} + b_n^\dagger e^{i \omega_n \tau} \right). \label{eq:bath_operator_tau}
\end{align}

Substituting \( B(\tau) \) and \( B(0) \) into Eq.~\eqref{eq:bath_correlator}:
\begin{equation} \label{eq:correlator_substitution}
C(\tau) = \left\langle \sum_{n=1}^{\infty} c_n \sqrt{\frac{1}{2 m_n \omega_n}} (b_n e^{-i \omega_n \tau} + b_n^\dagger e^{i \omega_n \tau}) \sum_{m=1}^{\infty} c_m \sqrt{\frac{1}{2 m_m \omega_m}} (b_m + b_m^\dagger) \right\rangle.
\end{equation}
Using the thermal expectation values:
\begin{align} \label{eq:thermal_expectations}
\langle b_n b_m^\dagger \rangle &= \delta_{nm} (n_n + 1), \quad \langle b_n^\dagger b_m \rangle = \delta_{nm} n_n,
\end{align}
where \( n_n \) is the Bose-Einstein distribution:
\begin{equation} \label{eq:bose_einstein_distribution}
n_n = \frac{1}{e^{\beta \omega_n} - 1}.
\end{equation}
We obtain:
\begin{equation} \label{eq:correlator_result}
C(\tau) = \sum_{n=1}^{\infty} \frac{c_n^2}{2 m_n \omega_n} \left[ (n_n + 1) e^{-i \omega_n \tau} + n_n e^{i \omega_n \tau} \right].
\end{equation}

\subsection{Spectral Density Representation}
\label{subsec:spectral_density}

Expressing \( C(\tau) \) in terms of the spectral density \( J(\omega) \), defined as:
\begin{equation} \label{eq:spectral_density}
J(\omega) = \pi \sum_{n=1}^{\infty} \frac{c_n^2}{2 m_n \omega_n} \delta(\omega - \omega_n),
\end{equation}
the bath correlator becomes:
\begin{equation} \label{eq:correlator_spectral_density}
C(\tau) = \int_0^\infty d\omega \frac{J(\omega)}{\pi} \left[ (n(\omega) + 1) e^{-i \omega \tau} + n(\omega) e^{i \omega \tau} \right].
\end{equation}
Rearranging terms:
\begin{equation} \label{eq:correlator_final}
C(\tau) = \int_0^\infty d\omega \frac{J(\omega)}{\pi} \left[ \coth\left( \frac{\beta \omega}{2} \right) \cos(\omega \tau) - i \sin(\omega \tau) \right].
\end{equation}
This is the final result for the bath correlator.

% Example of using todo notes
\chapter{Example of Todo Notes}

\section{Using Todo Notes}

\noindent Here's an example paragraph with various todo notes embedded. The Fourier transform is a mathematical transform that decomposes a function into its constituent frequencies. \todoimp{Complete this explanation with the formal definition!}

\noindent When discussing the Hamiltonian of a quantum system, we typically write it as:
\begin{equation}
    H = H_0 + H_{\text{int}}
\end{equation}

\todoeq{Add the interaction terms explicitly}

\noindent The spectroscopic signals in 2D spectroscopy contain peaks along the diagonal corresponding to the linear absorption spectrum. \todoref{Add reference to Mukamel's work}

\noindent In the phase-matching conditions section we could \todoidea{Add a diagram illustrating the k-vector directions in FWM experiments}

\noindent The derivation of the optical Bloch equations needs \todofix{Fix the coherence evolution equation - sign error in the term involving $\rho_{21}$}

\section{Summary of Todo Types}

\begin{itemize}
    \item \texttt{\\todoimp\{\}} - Red notes for high importance items
    \item \texttt{\\todoidea\{\}} - Green notes for ideas and suggestions
    \item \texttt{\\todoeq\{\}} - Blue notes for equations and mathematical content
    \item \texttt{\\todoref\{\}} - Purple notes for references and citations
    \item \texttt{\\todofix\{\}} - Orange notes for text that needs correction
\end{itemize}

% To get a list of all todos:
% \listoftodos[Notes \& Todos]

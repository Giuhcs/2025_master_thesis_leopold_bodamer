%----------------------------------------------------------------------------------------
%	PACKAGES AND OTHER DOCUMENT CONFIGURATIONS
%----------------------------------------------------------------------------------------
\documentclass[
	11pt, english, singlespacing,
%raft, % Uncomment to enable draft mode (no pictures, no links, overfull hboxes indicated)
%nolistspacing, % If the document is onehalfspacing or doublespacing, uncomment this to set spacing in lists to single
	liststotoc, % Uncomment to add the list of figures/tables/etc to the table of contents
	toctotoc, % Uncomment to add the main table of contents to the table of contents
%parskip, % Uncomment to add space between paragraphs
%nohyperref, % Uncomment to not load the hyperref package
%headsepline, % Uncomment to get a line under the header
%chapterinoneline, % Uncomment to place the chapter title next to the number on one line
%consistentlayout, % Uncomment to change the layout of the declaration, abstract and acknowledgements pages to match the default layout
]{style_thesis}

%----------------------------------------------------------------------------------------
%	PACKAGES
%----------------------------------------------------------------------------------------
\usepackage[utf8]{inputenc} % Required for inputting international characters
\usepackage[T1]{fontenc}
\usepackage{mathpazo} % Use the Palatino font by default

%-------------------------------- Define the LIBRARY ------------------------------------

\usepackage[backend=biber,style=numeric,sorting=none,sortcites=true,style=numeric-comp]{biblatex}
\AtBeginBibliography{\renewcommand*{\finalnamedelim}{\addspace\bibstring{and}\space}}
\renewcommand*{\bibnamedash}{\ifnumgreater{\value{liststop}}{1}{\space et al.}{}} % Et al. bei mehreren Autoren
%\renewcommand*{\bibfont}{\small} % Optional: Make the bibliography font size smaller
\addbibresource{latex/bib/my_bibliography.bib} % TODO load the bibliography, or how can I connect zotero?
\RequireBibliographyStyle{style_bibiliography}

%----------------------------------------------------------------------------------------


\usepackage[autostyle=true]{csquotes}

\usepackage{hyperref}
\usepackage{amsmath, amssymb, amsfonts}
\usepackage{bm}       % For bold Greek letters
\usepackage{caption}
\captionsetup{width=1.0\textwidth, font=normal, labelfont=bf}

%----------------------------------------------------------------------------------------
%	MARGIN SETTINGS
%----------------------------------------------------------------------------------------

\geometry{
	paper=a4paper,
	inner=2cm, % Inner margin
	outer=3cm, % Outer margin
	bindingoffset=.5cm, % Binding offset
	top=1.5cm, % Top margin
	bottom=1.5cm, % Bottom margin
	%showframe, % Uncomment to show how the type block is set on the page
}

%----------------------------------------------------------------------------------------
%	THESIS INFORMATION
%----------------------------------------------------------------------------------------

\thesistitle{Master thesis}
\supervisor{\href{https://ic1.ugr.es/members/dmanzano/home/}{Prof. Dr. Daniel Manzano Diosdado}}
\examiner{Prof. Dr. Beatriz \textsc{Olmos Sanchez}}
\degree{Master of Science}
\author{Leopold \textsc{Bodamer}}
\addresses{Wolfsbergallee 17b, 75177 Pforzheim}
\subject{Theoretical Atomic Physics and Synthetic Quantum Systems}
\universityone{\href{https://uni-tuebingen.de\\}{Eberhard Karls Universität Tübingen}}
\universitytwo{\href{https://www.ugr.es\\}{Universidad de Granada}}

\department{\href{https://uni-tuebingen.de/fakultaeten/mathematisch-naturwissenschaftliche-fakultaet/fachbereiche/physik/institute/institut-fuer-theoretische-physik/arbeitsgruppen/}{Institut für Theoretische Physik}}
\group{\href{https://uni-tuebingen.de/fakultaeten/mathematisch-naturwissenschaftliche-fakultaet/fachbereiche/physik/institute/institut-fuer-theoretische-physik/arbeitsgruppen/ag-lesanovsky/}{Theoretical Atomic Physics and Synthetic Quantum Systems}}
\faculty{\href{https://uni-tuebingen.de/fakultaeten/mathematisch-naturwissenschaftliche-fakultaet/fakultaet/}{Mathematisch-Naturwissenschaftliche Fakultät}}

\AtBeginDocument{
\hypersetup{pdftitle=\ttitle}
\hypersetup{pdfauthor=\authorname}
}

\begin{document}

\frontmatter
\pagestyle{plain}

% !TEX root = ../main.tex
%----------------------------------------------------------------------------------------
%	TITLE PAGE
%----------------------------------------------------------------------------------------

\begin{titlepage}
	\begin{center}

		%\vspace*{.06\textheight}
		{\scshape\LARGE \univone\\
			\&\par
			\univtwo\\
			\par}\vspace{1.5cm} % University name
		%\includegraphics[scale = 0.15]{Uni_logo}\vspace{1.5cm} % University/department logo - uncomment to place it

		\textsc{\Large Master Thesis}\\[0.5cm] % Thesis type

		\HRule \\[0.4cm] % Horizontal line
		{\huge \bfseries \ttitle\par}\vspace{0.4cm} % Thesis title

		\HRule \\[1.5cm] % Horizontal line

		\begin{minipage}[t]{0.4\textwidth}
			\begin{flushleft} \large
				\emph{Author:}\\
				{\authorname} % Author name
			\end{flushleft}
		\end{minipage}
		\begin{minipage}[t]{0.4\textwidth}
			\begin{flushright} \large
				\emph{Supervisor:} \\
				\href{https://sites.google.com/site/beatrizolmosphysics/}{\supname} % Supervisor name - remove the \href bracket to remove the link  
			\end{flushright}
		\end{minipage}\\[3cm]

		\vfill

		\large \textit{A thesis submitted in fulfillment of the requirements\\ for the degree of \degreename}\\[0.3cm] % University requirement text
		\textit{in }\\[0.4cm]
		\groupname\\\deptname\\[2cm] % Research group name and department name

		\vfill

		{\large \today}\\[4cm] % Date

		\vfill
	\end{center}
\end{titlepage}
%----------------------------------------------------------------------------------------
%	DECLARATION PAGE
%----------------------------------------------------------------------------------------

\begin{declaration}
	\addchaptertocentry{\authorshipname} % Add the declaration to the table of contents
	\noindent I, \authorname, declare that this thesis titled, \enquote{\ttitle} and the work presented in it are my own. I confirm that:

	\begin{itemize}
		\item This work was done wholly or mainly while applying for a research degree at this University.
		\item Where any part of this thesis has previously been submitted for a degree or any other qualification at this University or any other institution, this has been clearly stated.
		\item Where I have consulted the published work of others, this is always clearly attributed.
		\item Where I have quoted from the work of others, the source is always given. With the exception of such quotations, this thesis is entirely my own work.
		\item I have acknowledged all main sources of help.
		\item Where the thesis is based on work done by myself jointly with others, I have made clear exactly what was done by others and what I have contributed myself.
	\end{itemize}

	\noindent Signed:\\
	\rule[0.5em]{25em}{0.5pt} % This prints a line for the signature

	\noindent Date:\\
	\rule[0.5em]{25em}{0.5pt} % This prints a line to write the date
\end{declaration}

%----------------------------------------------------------------------------------------
%	ABSTRACT PAGE
%----------------------------------------------------------------------------------------

%\begin{abstract}
%\addchaptertocentry{\abstractname}
%This thesis investigates the directional routing of excitations in atomic systems using subradiant states.
%Building on Bottarelli's quantum router \cite{Startingpoint}, this thesis adapts the model to atomic systems,
%addressing the challenges of controlling interactions in fully connected systems.
%Atom light interactions have been heavily studied for atomic lattices \cite{Masson2022, Asenjo-Garcia2017, Needham2019, Cech2023, Jen2016}.
%In this theis three chains, that are connected by an equilateral triangle and an isosceles triangle are studied.
%By allowing different dipole orientations on each chain, three distinct topologies are considered.
%The results show that controlling the topology and the initial state enables directional routing,
%where a topology with equilateral triangle and aligned dipoles emerges as the most practical for stable readout.

	
%\end{abstract}
%----------------------------------------------------------------------------------------
%	CONTENTS
%----------------------------------------------------------------------------------------
\mainmatter
\pagestyle{latex/style/thesis}

\include{latex/chapters/c02_acknow_lists_abbrs_consts_symbls_dedication__tex} % Corrected chapter path
% !TEX root = ../main.tex
\chapter{Introduction} % Main chapter title
\label{Chapter_Introduction} % Change X to a consecutive number; for referencing this chapter elsewhere, use \ref{ChapterX}

%-------------------------------------------------------------------------------
%	SECTION 1: Coherence and Excitation Transport
%-------------------------------------------------------------------------------

\section{Coherence and Excitation Transport}

In this chapter, we aim to explain the phenomena of long coherences (lifetimes) and the excitation transport of light on a microtubule. The proposed model takes the following approach:

\begin{itemize}
	\item The microtubule is modeled as a cylindrical structure consisting of nodes. Each node represents an atom, which is modeled as a two-level system. The number of atoms, \( N_{\text{atoms}} \), is determined by the number of chains (\( n_{\text{chains}} \)) and the number of rings (\( n_{\text{rings}} \)), assuming fixed positions for these nodes.
	\item The system is restricted to a single excitation.
	\item A time-dependent coupling to an electric field is proposed, which may be either classical or quantum in nature. This coupling is intended to facilitate spectroscopy.
	\item Two types of Lindblad operators are introduced to model dissipation processes. Specifically:
	      \begin{enumerate}
		      \item Spontaneous decay
		      \item Dephasing
	      \end{enumerate}
\end{itemize}
The Lindblad operators introduced to model the spontaneous decay and dephasing processes for each individual atom are defined as follows:

\begin{align}
	C_{\text{decay}}^{(i)}   & = \sqrt{\gamma_0} \, \sigma_-^{(i)},    \\
	C_{\text{dephase}}^{(i)} & = \sqrt{\gamma_\phi} \, \sigma_z^{(i)},
\end{align}

where:
\begin{itemize}
	\item \( C_{\text{decay}}^{(i)} \) describes the spontaneous decay of the \(i\)-th atom, with a rate given by \(\gamma_0\).
	\item \( C_{\text{dephase}}^{(i)} \) describes the dephasing of the \(i\)-th atom, with a rate given by \(\gamma_\phi\).
	\item \( \sigma_-^{(i)} \) is the lowering operator for the \(i\)-th atom, and \( \sigma_z^{(i)} \) is the Pauli \( z \)-operator for the \(i\)-th atom.
\end{itemize}


\todoidea{Give a whole introduction to quantum biology, why it is interesting, quantum consciousness, ... , microtubules, coherence, excitation transport, 2D spectroscopy, with open quantum systems}
\newpage











\section{Motivation}
\noindent

%----------------------------------------------------------------------------------------
%	SECTION 1
%----------------------------------------------------------------------------------------
%\section{Objective}
%\vspace{0.5cm}
%\noindent
%The goal of this thesis is
%to perform robust directional photon routing on atomic systems in free-space using subradiant states.
%Focusing on a Y-shaped atomic tree, different topologies are explored to enable long-lived information transport as a proof of concept.
%
%\section{Outline}
%This thesis is structured as follows.
%Chapter \ref{Chapter2} introduces the theoretical background.
%It covers the concepts of open quantum systems,
%subradiance and superradiance, the Green tensor, and the reciprocal space.
%These tools are essential foundations for describing atom-atom interactions in free space,
%including dipole-dipole interactions and coupling to a photonic bath.
%%After this chapter, the reader already knows...
%The quantum router of \cite{startingpoint} is presented and summarized in Chapter \ref{Chapter3}.
%It introduces the concepts of graph theory and explains how quantum evolution on a graph topology can be utilized to achieve directional routing of information.
%Chapter \ref{Chapter4} will be the core of this thesis, adapting this model to an atomic system.
%This chapter delves into the challenges of implementing directional routing in a fully connected atomic system and investigates various solutions to control the phase of interactions.
%It further extends the analysis to systems with a larger number of atoms, focusing on coupling control and routing capabilities in different configurations, such as equilateral and isosceles triangles.
%Chapter \ref{Chapter5} concludes the thesis by summarizing the results and discussing potential future directions in the field of quantum routing in atomic systems.


It is widely assumed that one of the crucial tasks currently facing quantum theorists
is to understand and characterize the behaviour of realistic quantum systems. In
any experiment, a quantum system is subject to noise and decoherence due to the
unavoidable interaction with its surroundings. The theory of open quantum systems
aims at developing a general framework to analyze the dynamical behaviour of systems
that, as a result of their coupling with environmental degrees of freedom, will no
longer evolve unitarily. \cite{rivasetal2010markovianmasterequations}
\\
2DES> \todoref{rind ref}%\cite{krumlandetal2023twodimensionalelectronicspectroscopy}, 
\cite{segarra-martietal2018accuratesimulationtwodimensional}, \cite{sunetal2024twodimensionalspectroscopyopen}
\\
NONlinear Optics> \cite{hamm2005principlesnonlinearoptical}, \cite{mukamel1995principlesnonlinearoptical}
\\
Spectroscopy investigates the interaction between matter and electromagnetic radiation, offering a means to analyze composition and structure.
Central to this analysis is the understanding of how molecules respond to specific frequencies of light, revealing information about their energy levels and bonding.
Key concepts include wavelength ($\lambda$), wavenumber ($\bar{\nu}$), and frequency ($\nu$).
Wavelength, the distance between successive wave crests, is typically measured in nanometers or micrometers.
Wavenumber, expressed in inverse centimeters (cm$^{-1}$), represents the number of waves per unit distance and is directly proportional to energy, defined as $\bar{\nu} = 1/\lambda$ (where $\lambda$ is in cm).
Frequency, the number of wave cycles per second, is measured in Hertz (Hz), and the angular frequency ($\omega$) is related to frequency by $\omega = 2\pi\nu$.
The relationship between angular frequency and wavenumber is given by $\omega = 2\pi c \bar{\nu}$, where $c$ is the speed of light.\\
Next, I converted all units into femtoseconds ($fs^{-1}$), which is commonly used in time-domain spectroscopy.\\
Spectrometers are instruments designed to measure the intensity of light as a function of wavelength or frequency.\\
Different types of spectrometers are employed for various regions of the electromagnetic spectrum.
Notably, UV-Vis spectrometers analyze absorption and transmission of ultraviolet and visible light, while infrared (IR) spectrometers measure the absorption of infrared light, providing insights into molecular vibrations.
Nuclear Magnetic Resonance (NMR) spectrometers probe the magnetic properties of atomic nuclei, revealing molecular structure.

\chapter{Derivation of the Redfield Equation} % Main chapter title
\label{Chapter:Derivation_Redfield_Equation} % Label for referencing this chapter

The following derivation is part of \cite{manzanoShortIntroductionLindblad2020}
%----------------------------------------------------------------------------------------
%	SECTION 1
%----------------------------------------------------------------------------------------

\section{Derivation from microscopic dynamics}
\label{sec:Derivation_redfield_eq_from_microscopic_dynamics}
The most common derivation of the Redfield master equation is based on open quantum theory.
We will begin by discussing the general problem, where a small quantum system interacts with a larger environment.
The total system Hilbert space $\mathcal{H}_T$ is divided into our system of interest, belonging to a Hilbert space $\mathcal{H}_S$, and the environment living in $\mathcal{H}_E$.

\begin{equation}
	\mathcal{H}_T = \mathcal{H}_S \otimes \mathcal{H}_E.
	\label{eq:Total_Hilbert_Space}
\end{equation}
The Redfield equation is then an effective motion equation for a this subsystem of interest $ S $.
The derivation can be found in several textbooks such as Breuer and Petruccione \cite{breuerTheoryOpenQuantum2009}.

The evolution of the total system is given by the Liouville-Von Neumann equation, which represents the starting point of our derivation:

\begin{equation}
	\dot{\rho}_T(t) = -i[H_T, \rho_T(t)],
	\label{eq:Von_Neumann_Equation}
\end{equation}
where $\rho_T(t)$ is the density matrix of the total system, and $H_T$ is the total Hamiltonian.
Note that we used units where $\hbar = 1$.
This equation can not be solved for arbitrary large enviroments.
But as we are interested in the dynamics of the system without the environment,
we trace over the environment degrees of freedom to obtain the reduced density matrix of the system
$\rho(t) = \mathrm{Tr}_E[\rho_T]$.

The total Hamiltonian can be separated as:

\begin{equation}
	H_T = H_S \otimes \mathbb{1}_E + \mathbb{1}_S \otimes H_E + \alpha H_I,
	\label{eq:Total_Hamiltonian}
\end{equation}
where $H_S \in \mathcal{H}_S$, $H_E \in \mathcal{H}_E$, and $H_I \in \mathcal{H}_T$. $ H_I $ represents the interaction between the system and the environment with coupling strength $\alpha$.
The interaction term is typically decomposed as:

\begin{equation}
	H_I = \sum_i S_i \otimes E_i,
	\label{eq:Interaction_Hamiltonian}
\end{equation}
where $S_i \in \mathcal{B}(\mathcal{H}_S)$ and $E_i \in \mathcal{B}(\mathcal{H}_E)$ are operators, that only act on the system and environment respectively.

The following requirements are to be fuildilled by the final derived Redfield equation:

\begin{itemize}
	\item The equation should be linear in the system density matrix $ \dot{ \rho}_S(t) = F(\rho_S(t))$ (reduced equation of motion).
	\item The equation should be Markovian, meaning that the evolution of the system density matrix at time $t$ only depends on the state of the system at time $t$ and not on its past history $ \dot{ \rho}_S(t) = \rho_S(t)$ .
	\item The equation should be trace-preserving, meaning that $\mathrm{Tr}[\rho_S(t)] = \mathrm{Tr}[\rho_S(0)]$ for all times $t$.
\end{itemize}
Unlike the Linblad equation it does not guarantee the complete (not even normal) positivity of the density matrix, which is a requirement for a physical state.
So care has to be taken, when the Redfieldequation is useful.
The equation will be valid in the weak coupling limit, meaning that the constant in the interaction Hamiltonian $H_I$ fulfills  $\alpha \ll  1$.

%----------------------------------------------------------------------------------------
%	SUBSECTION 1
%----------------------------------------------------------------------------------------

\subsection{Interaction Picture}
\label{subsec:Interaction_Picture}

To describe the system dynamics, we move to the interaction picture where the operators evolve with respect to $H_S + H_E$.
Any arbitrary operator $O$ in the Schrödinger picture takes the form

\begin{equation}
	\hat{O}(t) = e^{i(H_S+H_E)t} O e^{-i(H_S+H_E)t},
	\label{eq:Interaction_Picture_Operators}
\end{equation}
in the interaction picture, and depends on time.
States now evolve only according to the interaction Hamiltonian $H_I$, and the Liouville-Von Neumann equation (Eq. \eqref{eq:Von_Neumann_Equation}) changes to:

\begin{equation}
	\dot{\hat{\rho}}_T(t) = -i \alpha [\hat{H}_I(t), \hat{\rho}_T(t)],
	\label{eq:LiouvilleVN}
\end{equation}
which can be formally integrated as:

\begin{equation}
	\hat{\rho}_T(t) = \hat{\rho}_T(0) - i \alpha \int_0^t ds [\hat{H}_I(s), \hat{\rho}_T(s)].
	\label{eq:Formal_Integration}
\end{equation}
This equation will be inserted in Eq. \eqref{eq:LiouvilleVN}:

\begin{equation}
	\dot{\hat{\rho}}_T(t) = -i \alpha \left[ \hat{H}_I(t), \hat{\rho}_T(0) \right]
	- \alpha^2 \int_0^t \left[ \hat{H}_I(t), \left[ \hat{H}_I(s), \hat{\rho}_T(s) \right] \right] ds,
	\label{eq:Second_Order_Expansion}
\end{equation}
which is not Morkovian, because of the intergal which sums up all the past.
The step of integration and insertion can be repreated leading to a series expansion of the density matrix in the small parameter $\alpha$:

\begin{equation}
	\dot{\hat{\rho}}_T(t) = -i \alpha \underbrace{\left[ \hat{H}_I(t), \hat{\rho}_T(0) \right]}_{\text{(1)}}
	- \alpha^2 \int_0^t \underbrace{\left[ \hat{H}_I(t), \left[ \hat{H}_I(s), \hat{\rho}_T(t) \right] \right]}_{\text{(2)}} ds + \mathcal{O} (\alpha^3).
	\label{eq:Second_Order_Expansion_wo_third}
\end{equation}
where third order contributions (and higher) are neglected from now on.
This can be justified in the weak coupling limit where $ \alpha \ll 1 $, which represents the "\textbf{Born}" approximation.

In this sense, the Redfield equation will be a second-order approximation of the actual dynamics.
The weak coupling assumption does not hold universally and cannot be applied to all systems.
For instance, it is often invalid in chemical or biological systems.

Remark, that Eq. \eqref{eq:Second_Order_Expansion_wo_third} is still not Morkovian, because of the intergal over time.
Since we are only interested in the dynamics of the system $ S $, we will now take the partial trace over the environment degrees of freedom in Eq. \eqref{eq:Second_Order_Expansion_wo_third}.

%----------------------------------------------------------------------------------------
%	SUBSECTION 2
%----------------------------------------------------------------------------------------

\subsection{Partial Trace}
\label{subsec:Partial_Trace}

We now assume the whole system to be seperatable at $ t = 0 $ as a product state:

\begin{equation}
	\hat{\rho}_T(0) = \hat{\rho}_S(0) \otimes \hat{\rho}_E(0),
	\label{eq:Initial_Product_State}
\end{equation}
which means, that the two sub-systems only come into contact at $ t = 0 $ and there are no correlations.
We take the interaction Hamiltonian Eq. \eqref{eq:Interaction_Hamiltonian}, which has the same shape in the Interaction Picture into account.
With this, the partial trace over the environment of the part (1) of Eq. \eqref{eq:Second_Order_Expansion_wo_third} is given by:

\begin{align}
	\sum_i \mathrm{Tr}_E\big[ S_i \otimes E_i , \hat{\rho}_S(0) \otimes \hat{\rho}_E(0)\big]
	= \sum_i \big(S_i \hat{\rho}_S(0) - \hat{\rho}_S(0) S_i\big) \cdot \mathrm{Tr}_E \big[E_i \hat{\rho}_E(0)\big],
	\label{eq:Trace_Relation_first_part}
\end{align}

where we used the cyclic property of the trace.
We define the average of the bath degrees of freedom at zero temperature:

\begin{equation}
	\langle E_i \rangle_0 \equiv \mathrm{Tr}_E \big[E_i \hat{\rho}_E(0)\big].
	\label{eq:Environment_Expectation_Value}
\end{equation}
which results to zero, simplifying Eq. \eqref{eq:Second_Order_Expansion_wo_third} to only the second part.
This can always be justified when adding a zero to the total Hamiltonian:

\begin{equation}
	H_T = H_S' + H_E + \alpha H_I',
	\label{eq:Shifted_Total_Hamiltonian}
\end{equation}
where a new interaction and system Hamiltonian are introduced.
The interaction Hamiltonian takes new environmental operators $E_i'$ which are shifted by the average of the environment operators at time $t = 0$:

\begin{equation}
	H_I' = \sum_i S_i \otimes E_i' = \sum_i S_i \otimes (E_i - \langle E_i \rangle_0).
	\label{eq:Shifted_Interaction_Hamiltonian}
\end{equation}
The new system Hamiltonian is then given by the sum of the original system Hamiltonian shifted proportionally by the average of the environment operators at time $t = 0$:

\begin{equation}
	H_S' = H_S + \alpha \sum_i S_i \langle E_i \rangle_0,
	\label{eq:Shifted_System_Hamiltonian}
\end{equation}
This however doesn't change the structure of the system dynamics.
It only accounts for a redefinition of the energy levels ("a sort of renormalization").
This way the equation \eqref{eq:Trace_Relation_first_part} results to zero and only the second part of the equation Eq. \eqref{eq:Second_Order_Expansion_wo_third} remains.

\begin{align}
	\dot{\rho}_S(t) = -i \alpha [\hat{H}_I(t),\hat{\rho}_T(0)]
	  & - \alpha^2 \int_0^t ds \mathrm{Tr}_E \big[\hat{H}_I(t), [\hat{H}_I(s), \rho_S(t) \otimes \rho_E]\big] \notag \\
	= & - \alpha^2 \int_0^t ds \mathrm{Tr}_E \big[\hat{H}_I(t), [\hat{H}_I(s), \rho_S(t) \otimes \rho_E]\big].
	\label{eq:Partial_Trace_Derivation}
\end{align}
In the follwoing, we will derive the final expression by calculating the environmental traces in the last equation.

%----------------------------------------------------------------------------------------
%	SUBSECTION 3
%----------------------------------------------------------------------------------------

\subsection{Final Expression}
\label{subsec:Final_Expression}

Defining $s' = t - s$, we rewrite the second-order term as:
\begin{align}
	\dot{\rho}_S(t)  = \alpha^2 \int_0^t ds \mathrm{Tr}_E \bigg\{
	\hat{H}_I(t) \big[ \hat{H}_I(t-s) \hat{\rho}_T(t) - \hat{\rho}_T(t) \hat{H}_I(t-s) \big] \notag \\
	- \big[ \hat{H}_I(t-s) \hat{\rho}_T(t) - \hat{\rho}_T(t) \hat{H}_I(t-s) \big] \hat{H}_I(t)
	\bigg\}.
	\label{eq:Second_Order_Final_Expression}
\end{align}
A seperatability at all times is now assumed:

\begin{equation}
	\hat{\rho}_T(t) = \hat{\rho}_S(t) \otimes \hat{\rho}_E(t),
	\label{eq:Reduced_Density_Matrix_Assumption}
\end{equation}
This assumtion has to be made even stronger later $\hat{\rho}_T(t) = \hat{\rho}_S(t) \otimes \hat{\rho}_E(0)$.
Expanding Eq. \eqref{eq:Second_Order_Final_Expression}, we obtain:

\begin{align}
	\dot{\hat{\rho}}_T(t) & =  \alpha^2 \int_0^t ds
	\bigg\{
	\mathrm{Tr}_E \big[ \hat{H}_I(t) \hat{H}_I(t-s) \hat{\rho}_T(t) \big] -
	\mathrm{Tr}_E \big[ \hat{H}_I(t) \hat{\rho}_T(t) \hat{H}_I(t-s) \big] - \notag \\
	                      & \qquad \qquad \qquad
	\mathrm{Tr}_E \big[ \hat{H}_I(t-s) \hat{\rho}_T(t) \hat{H}_I(t) \big] +
	\mathrm{Tr}_E \big[ \hat{\rho}_T(t) \hat{H}_I(t-s) \hat{H}_I(t) \big]
	\bigg\}.
	\label{eq:Expanded_Second_Order_Expression}
\end{align}

Now, inserting the interaction Hamiltonian by tracking the operators at time $t - s$ with $i'$ and at $t$ with $i$, we have:

\begin{align}
	\dot{\hat{\rho}}_T(t) & = \alpha^2  \sum_{i, i'} \int_0^t ds
	\bigg\{
	\mathrm{Tr}_E \big[ \hat{S}_i(t) \hat{S}_{i'}(t-s) \hat{\rho}_S(t)      \otimes   \hat{E}_{i}(t) \hat{E}_{i'}(t-s) \hat{\rho}_E(t)  \big] -  \notag                         \\
	                      & \mathrm{Tr}_E \big[ \hat{S}_i(t) \hat{\rho}_S(t) \hat{S}_{i'}(t-s)      \otimes   \hat{E}_{i}(t) \hat{\rho}_E(t) \hat{E}_{i'}(t-s)  \big] - \notag  \\
	                      & \mathrm{Tr}_E \big[ \hat{S}_{i'}(t-s) \hat{\rho}_S(t) \hat{S}_i(t)      \otimes   \hat{E}_{i'}(t-s) \hat{\rho}_E(t) \hat{E}_{i}(t)  \big] +  \notag \\
	                      & \mathrm{Tr}_E \big[ \hat{\rho}_S(t) \hat{S}_{i'}(t-s) \hat{S}_i(t)      \otimes   \hat{\rho}_E(t) \hat{E}_{i'}(t-s) \hat{E}_{i}(t)  \big]
	\bigg\}.
	\label{eq:Interaction_Hamiltonian_Expansion}
\end{align}

Since the trace only acts on the environment, the system operators can be taken out of the trace, and we define the correlation functions:
\begin{equation}
	C_{ij}(t - s) = \mathrm{Tr}_E \big[\hat{E}_{i}(t) \hat{E}_{i'}(t-s) \hat{\rho}_E(t)\big],
	\label{eq:Environment_Correlation_Function}
\end{equation}
such that:
\begin{align}
	\dot{\hat{\rho}}_T(t) = \alpha^2  \sum_{i, i'} \int_0^t ds
	\bigg\{
	C_{ij}(t - s) \big[ \hat{S}_i(t),  \hat{S}_{i'}(t-s) \hat{\rho}_S(t) \big] + \text{H.c.}
	\bigg\}.
	\label{eq:Redfield_Equation_Final}
\end{align}
which is the desired form of the Redfield equation.

Note however, that the we have not used the strong condition $\hat{\rho}_T(t) = \hat{\rho}_S(t) \otimes \hat{\rho}_E(0)$ was not used yet.
This however will make it possible to calculate the correlations Eq. \eqref{eq:Environment_Correlation_Function}, because we can assume that the enviroment is in a thermal equilibruium at a certain temperature.
Because of the assumption this is the case for all times.
It is equivalent to say that the environment is unaffected by the system. It is memoryless, because it is very big.

%----------------------------------------------------------------------------------------
%	APPENDICES
%----------------------------------------------------------------------------------------

\appendix

% !TEX root = ../main.tex
% Appendix Template

\chapter{Appendix Title Here} % Main appendix title

\label{AppendixX} % Change X to a consecutive letter; for referencing this appendix elsewhere, use \ref{AppendixX}

Write your Appendix content here. % corrected appendix path


%----------------------------------------------------------------------------------------
%	BIBLIOGRAPHY
%----------------------------------------------------------------------------------------

\printbibliography

\end{document}


% https://www.bing.com/images/search?view=detailV2&ccid=QOs8jh%2FY&id=B2F27160AAD4B68E1198585266845260137E1ED1&thid=OIP.QOs8jh_YfvbZty3qEQtmMwHaFv&mediaurl=https%3A%2F%2Fpubs.acs.org%2Fcms%2F10.1021%2Facs.jpcc.1c02693%2Fasset%2Fimages%2Flarge%2Fjp1c02693_0002.jpeg&cdnurl=https%3A%2F%2Fth.bing.com%2Fth%2Fid%2FR.40eb3c8e1fd87ef6d9b72dea110b6633%3Frik%3D0R5%252bE2BShGZSWA%26pid%3DImgRaw%26r%3D0&exph=1227&expw=1582&q=two+dimensional+spectroscopy+of+open+quantum+systems&simid=608020752819562198&form=IRPRST&ck=78FA3137380D45FD044E82D5CFD9637B&selectedindex=0&itb=0&ajaxhist=0&ajaxserp=0&vt=0&sim=11
%\begin{itemize}
%\item "Principles of Nonlinear Optical Spectroscopy" by Shan and Kirtman:
%\begin{itemize}
%\item This book provides a solid introduction to nonlinear optical spectroscopy, including 2D spectroscopy, and outlines methods for simulating spectroscopic signals from simple systems.
%\end{itemize}
%
%\item "Two-Dimensional Correlation Spectroscopy: Applications in Vibrational and Optical Spectroscopy" by R. M. B. L. McDonald and B. M. H. Lang:
%
%\begin{itemize}
%\item Focuses on 2D correlation spectroscopy and its applications in various spectroscopic fields, with simpler models used to illustrate the methods.
%\end{itemize}
%
%\item "Quantum Coherence in Systems of Atoms and Molecules: Theoretical and Experimental Studies" by A. Mukamel:
%
%\begin{itemize}
%\item This is a good resource for quantum descriptions of systems and how to simulate nonlinear spectroscopies like 2D spectroscopy in simple systems.
%\end{itemize}
%
%\item Review paper:
%
%\begin{itemize}
%\item \textit{"Two-Dimensional Electronic Spectroscopy"}, by A. M. Walmsley, Y. O. I. Shim, et al. (2009).\begin{itemize}
%\item A very detailed review of 2D electronic spectroscopy, including its application to simple systems, methods of analysis, and computational approaches.
%\end{itemize}
%
%\end{itemize}
%
%\end{itemize}
%\subsection*{2. Scientific Papers with Simulations of Simple Systems}
%\begin{itemize}
%\item "Two-dimensional electronic spectroscopy of molecular aggregates" by J. M. Thomas, D. V. Voronine, et al. (2013):
%
%\begin{itemize}
%\item Provides a step-by-step guide to performing 2D spectroscopy on simple systems like molecular aggregates. This paper discusses both the theoretical framework and simulations.
%\end{itemize}
%
%\item "Quantum Coherent Effects in Two-Dimensional Electronic Spectroscopy of Model Systems" by Y. Yan, A. Aspuru-Guzik, et al. (2010):
%
%\begin{itemize}
%\item Focuses on the theoretical treatment of 2D electronic spectroscopy of small systems, including simple chromophores and model molecular systems.
%\end{itemize}
%
%\item "Two-Dimensional Fourier Transform Spectroscopy of Single Chromophores" by D. V. Voronine, et al. (2012):
%
%\begin{itemize}
%\item The paper discusses the application of 2D spectroscopy to individual chromophores (a simple system) and explains how to extract physical information from the experimental data.
%\end{itemize}
%
%\end{itemize}
%\subsection*{3. Computational Methods and Software}
%If you're interested in simulating 2D electronic spectroscopy numerically for simple systems, the following computational tools are often used:
%
%\begin{itemize}
%\item PySCF (Python for Strongly Correlated Electron Systems):
%
%\begin{itemize}
%\item A computational chemistry package that can simulate molecular systems and might help you calculate the response functions used in 2D spectroscopy.
%\end{itemize}
%
%\item Qutip (Quantum Toolbox in Python):
%
%\begin{itemize}
%\item Useful for simulating open quantum systems, calculating time evolution, and modeling processes like those in 2D electronic spectroscopy.
%\end{itemize}
%
%\item TDSCF (Time-Dependent Self-Consistent Field Theory):
%
%\begin{itemize}
%\item A method often used in simulations of 2D electronic spectroscopy, this software may help in constructing a basic model for your system.
%\end{itemize}
%
%\item 2D NMR (for Nuclear Systems):
%
%\begin{itemize}
%\item While this focuses on nuclear magnetic resonance, many of the methods developed for 2D NMR spectroscopy can also be adapted to electronic spectroscopy, and relevant software tools include NMRPipe or MestReNova.
%\end{itemize}
%
%\end{itemize}
%\subsection*{4. Additional Resources}
%\begin{itemize}
%\item Tutorials and Lecture Notes:\begin{itemize}
%\item Several universities offer courses on nonlinear spectroscopy and 2D electronic spectroscopy. Look for lecture notes or course material available online, which can provide step-by-step methods to simulate simple systems. For example, the MIT OpenCourseWare site often has useful resources for quantum and optical spectroscopy.
%\end{itemize}
%
%\item Theoretical Approaches:\begin{itemize}
%\item "Nonlinear Optical Spectroscopy" by O. L. Chapman: This book will provide you with insights into the theory and mathematical foundations of 2D electronic spectroscopy for simple molecular systems.
%\item "Nonlinear Optics" by Robert W. Boyd: Boyd's book is a widely regarded source for nonlinear optics, including 2D spectroscopy, and covers applications to simpler systems such as isolated molecules or aggregates.
%\end{itemize}
%
%\end{itemize}
